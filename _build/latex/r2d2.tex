%% Generated by Sphinx.
\def\sphinxdocclass{jsbook}
\documentclass[letterpaper,10pt,dvipdfmx]{sphinxmanual}
\ifdefined\pdfpxdimen
   \let\sphinxpxdimen\pdfpxdimen\else\newdimen\sphinxpxdimen
\fi \sphinxpxdimen=.75bp\relax

\PassOptionsToPackage{warn}{textcomp}


\usepackage{cmap}
\usepackage[T1]{fontenc}
\usepackage{amsmath,amssymb,amstext}



\usepackage{times}


\usepackage{sphinx}

\fvset{fontsize=\small}
\usepackage[dvipdfm]{geometry}

% Include hyperref last.
\usepackage{hyperref}
% Fix anchor placement for figures with captions.
\usepackage{hypcap}% it must be loaded after hyperref.
% Set up styles of URL: it should be placed after hyperref.
\urlstyle{same}

\usepackage{sphinxmessages}
\setcounter{tocdepth}{1}



\title{R2D2}
\date{2019年12月20日}
\release{1.1}
\author{Hideyuki Hotta}
\newcommand{\sphinxlogo}{\vbox{}}
\renewcommand{\releasename}{リリース}
\makeindex
\begin{document}

\pagestyle{empty}
\sphinxmaketitle
\pagestyle{plain}
\sphinxtableofcontents
\pagestyle{normal}
\phantomsection\label{\detokenize{index::doc}}


このページは太陽のための輻射磁気流体コードR2D2(RSST and Radiation for Deep Dynamics)
のマニュアルである。

\noindent\sphinxincludegraphics[width=350\sphinxpxdimen]{{R2D2_logo}.png}


\chapter{とりあえず始めるには}
\label{\detokenize{start:id1}}\label{\detokenize{start::doc}}

\chapter{数値スキーム}
\label{\detokenize{scheme:id1}}\label{\detokenize{scheme::doc}}

\section{MHDスキーム}
\label{\detokenize{scheme:mhd}}

\subsection{空間微分}
\label{\detokenize{scheme:id2}}
R2D2では、4次の中央差分を用いている。格子間隔が一様な場合には中央差分では微分は
\begin{equation*}
\begin{split}\left(\frac{\partial q}{\partial x}\right)_i =\frac{-q_{i+2}+8q_{i+1}-8q_{i-1}+q_{i-2}}{12\Delta x_i}\end{split}
\end{equation*}
となる。R2D2では、非一様な格子間隔にも対応しており、


\subsection{時間積分}
\label{\detokenize{scheme:id3}}
R2D2では、


\chapter{フリーパラメタ}
\label{\detokenize{parameter:id1}}\label{\detokenize{parameter::doc}}

\chapter{方程式}
\label{\detokenize{equation:id1}}\label{\detokenize{equation::doc}}

\chapter{コードの構造}
\label{\detokenize{code:id1}}\label{\detokenize{code::doc}}

\chapter{座標生成}
\label{\detokenize{geometry:id1}}\label{\detokenize{geometry::doc}}

\chapter{境界条件}
\label{\detokenize{boundary:id1}}\label{\detokenize{boundary::doc}}
論文を書くときは
\begin{itemize}
\item {} 
\(x\) , \(y\) : 水平方向

\item {} 
\(z\) : 鉛直方向

\end{itemize}

となっているが、R2D2のコード内では
\begin{itemize}
\item {} 
\sphinxcode{\sphinxupquote{x}}: 鉛直方向

\item {} 
\sphinxcode{\sphinxupquote{y}}, \sphinxcode{\sphinxupquote{z}}: 水平方向

\end{itemize}

となっている。この取扱説明書では、コードに合わせた表記を用いる。

また、対称・反対称とは以下のような状況を表す。

\noindent\sphinxincludegraphics[width=450\sphinxpxdimen]{{bc_sym}.png}


\section{上部境界}
\label{\detokenize{boundary:id2}}

\subsection{ポテンシャル磁場}
\label{\detokenize{boundary:id3}}
磁場があるときは、上部ではポテンシャル磁場境界条件を使う。


\section{下部境界}
\label{\detokenize{boundary:id4}}
開く時
どの質量フラックスも対称にする。 計算領域内での質量を一定に保つために、水平に平均した密度
\(\langle \rho_1\rangle\) は反対称。 そこからのずれ
\(\rho_1 - \langle \rho_1 \rangle\) は対称な境界条件をとる。

一方、エントロピー \(s_1\) は上昇流で反対称、下降流で反対称な境界条件をとる。
この心は、開く境界条件を取るときは計算をしている領域の結果は信用するが、外から入ってくる物理量は、
計算領域に寄らないというものである。 下降流は現在計算している領域内部での情報を持って計算領域の外に出ていくので、
対称な境界条件を用いる。一方、上昇流は、計算している領域の外からの情報を持って計算領域に入ってくるので、
反対称な境界条件を用いて擾乱をゼロにする。これは元々のModel Sでの量を上昇流のエントロピーに用いるということである。


\chapter{人工粘性}
\label{\detokenize{artdif:id1}}\label{\detokenize{artdif::doc}}

\chapter{入力・出力}
\label{\detokenize{io:id1}}\label{\detokenize{io::doc}}

\section{バージョン履歴}
\label{\detokenize{io:id2}}\begin{itemize}
\item {} 
ver. 1.0: バージョン制を導入

\item {} 
ver. 1.1: 光学的厚さが0.1, 0.01の部分も出力することにした。qq\_in, vcをconfigのグローバル変数として取扱ことにした。

\end{itemize}


\chapter{Sphinx使用の覚書}
\label{\detokenize{sphinx:sphinx}}\label{\detokenize{sphinx::doc}}

\section{はじめに}
\label{\detokenize{sphinx:id1}}
Sphinxは、reStructuredTextからHTMLやLatexなどの
文章を生成するソフトウェアである。
\sphinxhref{https://www.sphinx-doc.org/ja/master/index.html}{Sphinxの公式サイト}
最近ではMarkdownでも記述できるが、結局最後のところはreStructuredTextで記述することになるので、現状では、Markdownは使用していない。このウェブサイトもSphinxで生成しているので、覚書をここに記す。


\section{インストール}
\label{\detokenize{sphinx:id3}}
ここではAnacondaがすでにインストールしてあるMac
にSphinxをインストールすることを考える。
基本的には以下のコマンドを実行するのみである。

\begin{sphinxVerbatim}[commandchars=\\\{\}]
pip install sphinx
\end{sphinxVerbatim}

Markdownを使いたい時は以下のようにする。

\begin{sphinxVerbatim}[commandchars=\\\{\}]
pip install commonmark recommonmark
\end{sphinxVerbatim}


\section{HTMLファイルの生成}
\label{\detokenize{sphinx:html}}
適当なディレクトリを作成(ここでは \sphinxcode{\sphinxupquote{test}} )とする。
そこで、 \sphinxcode{\sphinxupquote{sphinx-quickstart}} コマンドによりSphinxで作るドキュメントの
初期設定を行う。

\begin{sphinxVerbatim}[commandchars=\\\{\}]
mkdir \PYG{n+nb}{test} \PYG{c+c1}{\PYGZsh{} ディレクトリ作成}
\PYG{n+nb}{cd} \PYG{n+nb}{test}    \PYG{c+c1}{\PYGZsh{} ディレクトリに移動}
sphinx\PYGZhy{}quickstart
\end{sphinxVerbatim}

いくつか質問をされる。基本的には読めばわかる質問であるが
少し戸惑う質問を以下にあげる。
\begin{itemize}
\item {} 
プロジェクトのリリース: 1.0などとversionを答える。後に \sphinxcode{\sphinxupquote{conf.py}} を編集すれば変更可能

\item {} 
プロジェクトの言語: デフォルトは英語の \sphinxcode{\sphinxupquote{en}} であるが、日本語を使いたい時は \sphinxcode{\sphinxupquote{ja}} とする

\end{itemize}


\section{VS codeの利用}
\label{\detokenize{sphinx:vs-code}}
VS codeを利用すると快適にreStructuredTextを作成することができる。
\sphinxcode{\sphinxupquote{*.rst}} ファイルをVS codeで開くと自動で確認されるが、以下のプラグインをインストールする。

\noindent\sphinxincludegraphics[width=500\sphinxpxdimen]{{restructuredtext_vs}.png}

\sphinxcode{\sphinxupquote{Cmd+k Cmd+r}} で画面を分割してプレビューできる。正しい \sphinxcode{\sphinxupquote{conf.py}} の場所を設定する必要がある。


\section{環境設定}
\label{\detokenize{sphinx:id4}}
デフォルトの設定では、数式を書く時にMathjaxを使用するようで、数式の太字が意図するように表示されなかったのでsvgで出力することにした。
以下のように \sphinxcode{\sphinxupquote{conf.py}} に追記する。

\begin{sphinxVerbatim}[commandchars=\\\{\}]
\PYG{n}{extensions} \PYG{o}{+}\PYG{o}{=} \PYG{p}{[}\PYG{l+s+s1}{\PYGZsq{}}\PYG{l+s+s1}{sphinx.ext.imgmath}\PYG{l+s+s1}{\PYGZsq{}}\PYG{p}{]}
\PYG{n}{imgmath\PYGZus{}image\PYGZus{}format} \PYG{o}{=} \PYG{l+s+s1}{\PYGZsq{}}\PYG{l+s+s1}{svg}\PYG{l+s+s1}{\PYGZsq{}}
\PYG{n}{pngmath\PYGZus{}latex}\PYG{o}{=}\PYG{l+s+s1}{\PYGZsq{}}\PYG{l+s+s1}{platex}\PYG{l+s+s1}{\PYGZsq{}}
\end{sphinxVerbatim}

また、ウェブサイトのテーマを変更することもできる。どのようなテーマがあるかは
\sphinxhref{https://sphinx-users.jp/cookbook/changetheme/index.html}{Sphinxのテーマ}
を参照。好きなテーマを選んで \sphinxcode{\sphinxupquote{conf.py}} に以下のように設定。

\begin{sphinxVerbatim}[commandchars=\\\{\}]
\PYG{n}{html\PYGZus{}theme} \PYG{o}{=} \PYG{l+s+s1}{\PYGZsq{}}\PYG{l+s+s1}{bizstyle}\PYG{l+s+s1}{\PYGZsq{}}
\PYG{n}{html\PYGZus{}theme\PYGZus{}options} \PYG{o}{=} \PYG{p}{\PYGZob{}}\PYG{l+s+s1}{\PYGZsq{}}\PYG{l+s+s1}{maincolor}\PYG{l+s+s1}{\PYGZsq{}} \PYG{p}{:} \PYG{l+s+s2}{\PYGZdq{}}\PYG{l+s+s2}{\PYGZsh{}696969}\PYG{l+s+s2}{\PYGZdq{}}\PYG{p}{\PYGZcb{}}
\end{sphinxVerbatim}

今後変更の余地あり。


\section{記法}
\label{\detokenize{sphinx:id6}}

\subsection{リンク}
\label{\detokenize{sphinx:id7}}

\subsection{コード}
\label{\detokenize{sphinx:id8}}

\subsection{画像}
\label{\detokenize{sphinx:id9}}

\subsection{数式}
\label{\detokenize{sphinx:id10}}

\chapter{ライセンス}
\label{\detokenize{index:id1}}
R2D2は現状、公開ソフトウェアではなく再配布も禁じている。
これは今後、変更される可能性はあるが、開発者(堀田)の利益を守るためである。
共同研究者のみが使って良いというルールになっており、R2D2の使用には、以下の規約を守る必要がある。
\begin{itemize}
\item {} 
再配布しない

\item {} 
改変は許されるが、その時の実行結果については開発者は責任を持たない

\item {} 
R2D2で実行する計算は、堀田と議論する必要がある。パラメタ変更などの細かい変更には相談する必要はないが、新しいプロジェクトを開始するときはその都度相談すること。堀田自身のプロジェクト、堀田の指導学生のプロジェクトとの重複を避けるためである。

\item {} 
R2D2を用いた論文を出版するときは \sphinxhref{https://ui.adsabs.harvard.edu/abs/2019SciA....5.2307H/abstract}{Hotta et al., 2019} を引用すること。より詳しく説明した論文も出版予定である。

\end{itemize}


\chapter{索引と検索ページ}
\label{\detokenize{index:id2}}\begin{itemize}
\item {} 
\DUrole{xref,std,std-ref}{genindex}

\item {} 
\DUrole{xref,std,std-ref}{search}

\end{itemize}



\renewcommand{\indexname}{索引}
\printindex
\end{document}