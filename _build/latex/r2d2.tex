%% Generated by Sphinx.
\def\sphinxdocclass{jsbook}
\documentclass[letterpaper,10pt,dvipdfmx,report]{sphinxmanual}
\ifdefined\pdfpxdimen
   \let\sphinxpxdimen\pdfpxdimen\else\newdimen\sphinxpxdimen
\fi \sphinxpxdimen=.75bp\relax

\PassOptionsToPackage{warn}{textcomp}


\usepackage{cmap}
\usepackage[T1]{fontenc}
\usepackage{amsmath,amssymb,amstext}



\usepackage{times}


\usepackage{sphinx}

\fvset{fontsize=\small}
\usepackage[dvipdfm]{geometry}


% Include hyperref last.
\usepackage{hyperref}
% Fix anchor placement for figures with captions.
\usepackage{hypcap}% it must be loaded after hyperref.
% Set up styles of URL: it should be placed after hyperref.
\urlstyle{same}

\usepackage{sphinxmessages}
\setcounter{tocdepth}{1}



\title{R2D2}
\date{2020年04月28日}
\release{1.2}
\author{Hideyuki Hotta}
\newcommand{\sphinxlogo}{\vbox{}}
\renewcommand{\releasename}{リリース}
\makeindex
\begin{document}

\pagestyle{empty}
\sphinxmaketitle
\pagestyle{plain}
\sphinxtableofcontents
\pagestyle{normal}
\phantomsection\label{\detokenize{index::doc}}


このページは太陽のための輻射磁気流体コードR2D2(RSST and Radiation for Deep Dynamics)
のマニュアルである。

\sphinxhref{https://github.com/hottahd/R2D2-manual/raw/master/\_build/latex/r2d2.pdf}{PDF版はこちら}

\noindent\sphinxincludegraphics[width=350\sphinxpxdimen]{{R2D2_logo_red}.png}


\chapter{R2D2を使い始めるには}
\label{\detokenize{start:r2d2}}\label{\detokenize{start::doc}}

\section{ディレクトリ構造の準備}
\label{\detokenize{start:id1}}
R2D2は現在公開していないので、R2D2のzipファイルを堀田から受け取ったと仮定する。コードをしっかり読めば従う必要はないが、基本的には以下のようなディレクトリ構造で計算することを想定している。

\begin{sphinxVerbatim}[commandchars=\\\{\}]
project\PYGZus{}name/
           ├─ run/
           │    ├─ d001/
           │    ├─ d002/
           │    ├─ d003/
           │    ├─ ...
           │
           ├─ py/
           └─ idl/
\end{sphinxVerbatim}

pyには \sphinxhref{https://github.com/hottahd/R2D2\_py}{R2D2\_py} をクローンしてきたもの、idlには \sphinxhref{https://github.com/hottahd/R2D2\_idl}{R2D2\_idl} をクローンしてきたものを配置する。ここは名前が変わっても問題ない。python(py)とidlのどちらかを使えば解析は可能である(両方ダウンロードする必要はない)。

堀田から受け取ったR2D2.zipファイルをそれぞれrunディレクトリのd001などと名前を変えて配置することで色々な計算ケースを実行するのが良いだろう。

R2D2ディレクトリの中は以下のようなディレクトリ構造になっている。

\begin{sphinxVerbatim}[commandchars=\\\{\}]
R2D2/
   ├─ F90\PYGZus{}deps.py
   ├─ Makefile
   ├─ README.md
   ├─ gen\PYGZus{}time.py
   ├─ data/
   │     ├─ param/
   │     │      ├─ nd.dac
   │     │      ├─ back.dac
   │     │      ├─ params.dac
   │     │      └─ xyz.dac
   │     │
   │     ├─ qq/
   │     │   ├─ qq.dac.e
   │     │   ├─ qq.dac.o
   │     │   ├─ qq.dac.00000000
   │     │   ├─ qq.dac.00000001
   │     │   ├─ ...
   │     │
   │     ├─ tau/
   │     │   ├─ qq.dac.00000000
   │     │   ├─ qq.dac.00000001
   │     │   ├─ ...
   │     │
   │     ├─ remap/
   │     │      ├─ remap\PYGZus{}info.dac
   │     │      ├─ qq/
   │     │      │   ├─ qq.dac.00000000
   │     │      │   ├─ qq.dac.00000001
   │     │      │   ├─ ...
   │     │      │
   │     │      └─ vl/
   │     │          ├─ c.dac
   │     │          ├─ vla.dac.00000000
   │     │          ├─ vla.dac.00000001
   │     │          ├─ vla.dac.00000002
   │     │          ├─ ...
   │     │
   │     └─ time/
   │           ├─ mhd/
   │           │    ├─ t.dac.00000000
   │           │    ├─ t.dac.00000001
   │           │    ├─ ...
   │           │
   │           └─ tau/
   │                ├─ t.dac.00000000
   │                ├─ t.dac.00000001
   │                ├─ ...
   │
   ├─ input\PYGZus{}data/
   ├─ make/
   ├─ retired\PYGZus{}src/
   ├─ sh/
   └─ src/
        ├─ all/
        └─ include/
\end{sphinxVerbatim}

それぞれのファイル、ディレクトリの簡単な説明は以下である。
\begin{itemize}
\item {} \begin{description}
\item[{F90\_deps.py}] \leavevmode
make/Makefile生成のためのpythonスクリプト。fortranコードの依存性を調べて、make/R2D2.depsに出力する。新しいプログラムを作成したときは

\begin{sphinxVerbatim}[commandchars=\\\{\}]
python F90\PYGZus{}deps.py
\end{sphinxVerbatim}

としてmake/R2D2.depsを更新する。

\end{description}

\item {} \begin{description}
\item[{Makefile}] \leavevmode
makeをするときにmakeディクレクトリに移動する為のファイル。編集する必要はない。編集すべきMakefileはmake/Makefileに配置してある。

\end{description}

\item {} \begin{description}
\item[{README.md}] \leavevmode
GitHubに表示する為の説明ファイル。情報が古くなっている可能性があるので、README.mdを見るよりは、このウェブページの情報を参照されたい。

\end{description}

\item {} \begin{description}
\item[{gen\_time.py}] \leavevmode
他のモデルから計算結果をもらったときにdataディレクトリや時間のファイルを生成する為のpythonスクリプト

\end{description}

\item {} \begin{description}
\item[{copy\_caseid.py}] \leavevmode
続きの計算を行うためのスクリプト

\end{description}

\item {} \begin{description}
\item[{data/}] \leavevmode
fortranプログラムを実行した後に、データが保存されるディレクトリ。fortranプログラムを実行すると自動的に生成される。
\begin{itemize}
\item {} \begin{description}
\item[{data/param/}] \leavevmode
基本的な計算設定パラメタを出力する為のディレクトリ

\end{description}

\item {} \begin{description}
\item[{data/qq/}] \leavevmode
チェックポイントのための3次元データを出力するためのディレクトリ

\end{description}

\item {} \begin{description}
\item[{data/remap/}] \leavevmode
解析のためのリマッピングをしたあとのデータを格納するディレクトリ
\begin{itemize}
\item {} \begin{description}
\item[{data/remap/qq/}] \leavevmode
計算データをリマッピングして扱いやすくした三次元データ。単精度で出力。解析には主にこのデータを使う。

\end{description}

\item {} \begin{description}
\item[{data/remap/vl/}] \leavevmode
計算実行中の解析データ格納

\end{description}

\end{itemize}

\end{description}

\item {} \begin{description}
\item[{data/time/}] \leavevmode
出力した時間を記録したファイルを格納するディレクトリ。3次元データの出力の時間を記録する \sphinxcode{\sphinxupquote{mhd}} と光学的厚さ一定の場所の出力の時間を記録する \sphinxcode{\sphinxupquote{tau}} のディレクトリがある。
\begin{itemize}
\item {} \begin{description}
\item[{data/time/mhd/}] \leavevmode
MHD量のアウトプットの時間データ

\end{description}

\item {} \begin{description}
\item[{data/time/tau/}] \leavevmode
光学的厚さ一定の面アウトプットの時間データ

\end{description}

\end{itemize}

\end{description}

\end{itemize}

\end{description}

\end{itemize}


\section{コンパイル}
\label{\detokenize{start:id2}}
コンパイルは基本的にR2D2のディクレトリで

\begin{sphinxVerbatim}[commandchars=\\\{\}]
make
\end{sphinxVerbatim}

とするのみである。使う計算機によって設定が違うので \sphinxcode{\sphinxupquote{make/Makfile}} を編集する必要がある。
このファイルの1行目に

\begin{sphinxVerbatim}[commandchars=\\\{\}]
\PYG{n+nv}{SERVER}\PYG{o}{=}OFP
\end{sphinxVerbatim}

などと書いてある部分がある。計算機に応じてこの部分を書き換える。
それぞれ以下のような対応になっている。すでに使用できない計算機については説明しない。
\begin{itemize}
\item {} 
\sphinxcode{\sphinxupquote{XC}}: CfCA XC50

\item {} 
\sphinxcode{\sphinxupquote{OFP}}: Oakforest\sphinxhyphen{}PACS

\item {} 
\sphinxcode{\sphinxupquote{FX}}: 名大FX100

\end{itemize}

以下、堀田の個人環境なので、使用は推奨されない。どうしても個人の環境で使いたい時は堀田まで相談されたい。
\begin{itemize}
\item {} 
\sphinxcode{\sphinxupquote{LOCAL}}: UbuntuのGCC

\item {} 
\sphinxcode{\sphinxupquote{LOXAL\_ifort}}: Ubuntuのifort

\item {} 
\sphinxcode{\sphinxupquote{MAC}}: MacのGCC

\end{itemize}


\section{基本的なパラメータ}
\label{\detokenize{start:id3}}
主に変更するパラメタは、\sphinxstylestrong{領域サイズ} と \sphinxstylestrong{格子点数} であろう。

これらは、\sphinxcode{\sphinxupquote{src/all/geometry\_def.F90}} を編集することで変更できる。

まずは領域サイズ

\begin{sphinxVerbatim}[commandchars=\\\{\}]
\PYG{k+kt}{real}\PYG{p}{(}\PYG{n+nb}{KIND}\PYG{p}{(}\PYG{l+m+mf}{0.d0}\PYG{p}{)}\PYG{p}{)}\PYG{p}{,} \PYG{k}{parameter} \PYG{k+kd}{::} \PYG{n}{xmin} \PYG{o}{=} \PYG{n}{rsun} \PYG{o}{\PYGZhy{}} \PYG{l+m+mi}{2}\PYG{l+m+mf}{3.876d8}
\PYG{k+kt}{real}\PYG{p}{(}\PYG{n+nb}{KIND}\PYG{p}{(}\PYG{l+m+mf}{0.d0}\PYG{p}{)}\PYG{p}{)}\PYG{p}{,} \PYG{k}{parameter} \PYG{k+kd}{::} \PYG{n}{xmax} \PYG{o}{=} \PYG{n}{rsun} \PYG{o}{+} \PYG{l+m+mf}{0.7d8}
\PYG{k+kt}{real}\PYG{p}{(}\PYG{n+nb}{KIND}\PYG{p}{(}\PYG{l+m+mf}{0.d0}\PYG{p}{)}\PYG{p}{)}\PYG{p}{,} \PYG{k}{parameter} \PYG{k+kd}{::} \PYG{n}{ymin} \PYG{o}{=} \PYG{l+m+mf}{0.d0}
\PYG{k+kt}{real}\PYG{p}{(}\PYG{n+nb}{KIND}\PYG{p}{(}\PYG{l+m+mf}{0.d0}\PYG{p}{)}\PYG{p}{)}\PYG{p}{,} \PYG{k}{parameter} \PYG{k+kd}{::} \PYG{n}{ymax} \PYG{o}{=} \PYG{l+m+mf}{6.144d8}\PYG{o}{*}\PYG{l+m+mi}{1}\PYG{l+m+mf}{6.d0}
\PYG{k+kt}{real}\PYG{p}{(}\PYG{n+nb}{KIND}\PYG{p}{(}\PYG{l+m+mf}{0.d0}\PYG{p}{)}\PYG{p}{)}\PYG{p}{,} \PYG{k}{parameter} \PYG{k+kd}{::} \PYG{n}{zmin} \PYG{o}{=} \PYG{l+m+mf}{0.d0}
\PYG{k+kt}{real}\PYG{p}{(}\PYG{n+nb}{KIND}\PYG{p}{(}\PYG{l+m+mf}{0.d0}\PYG{p}{)}\PYG{p}{)}\PYG{p}{,} \PYG{k}{parameter} \PYG{k+kd}{::} \PYG{n}{zmax} \PYG{o}{=} \PYG{l+m+mf}{6.144d8}\PYG{o}{*}\PYG{l+m+mi}{1}\PYG{l+m+mf}{6.d0}
\end{sphinxVerbatim}

と書いてある箇所で領域サイズを決定している。{\color{red}\bfseries{}*}min, {\color{red}\bfseries{}*}maxはそれぞれ、{\color{red}\bfseries{}*}方向の領域の最小値、最大値である。x方向については、太陽中心からの距離で定義してあるので、 \sphinxcode{\sphinxupquote{rsun}} を使うのが推奨される。

次に格子点数

\begin{sphinxVerbatim}[commandchars=\\\{\}]
\PYG{k+kt}{integer}\PYG{p}{,} \PYG{k}{parameter}\PYG{p}{,} \PYG{k}{private} \PYG{k+kd}{::} \PYG{n}{nx0} \PYG{o}{=} \PYG{l+m+mi}{128}\PYG{p}{,} \PYG{n}{ny0} \PYG{o}{=} \PYG{l+m+mi}{64}\PYG{p}{,} \PYG{n}{nz0} \PYG{o}{=} \PYG{l+m+mi}{64}
\end{sphinxVerbatim}

\begin{sphinxVerbatim}[commandchars=\\\{\}]
\PYG{k+kt}{integer}\PYG{p}{,} \PYG{k}{parameter} \PYG{k+kd}{::} \PYG{n}{ix0} \PYG{o}{=} \PYG{l+m+mi}{4}
\PYG{k+kt}{integer}\PYG{p}{,} \PYG{k}{parameter} \PYG{k+kd}{::} \PYG{n}{jx0} \PYG{o}{=} \PYG{l+m+mi}{16}
\PYG{k+kt}{integer}\PYG{p}{,} \PYG{k}{parameter} \PYG{k+kd}{::} \PYG{n}{kx0} \PYG{o}{=} \PYG{l+m+mi}{16}
\end{sphinxVerbatim}

などと書いてある箇所がある。\sphinxcode{\sphinxupquote{nx0}} , \sphinxcode{\sphinxupquote{ny0}} , \sphinxcode{\sphinxupquote{nz0}} はそれぞれ一つのMPIプロセスでのx, y, z方向の格子点の数を定義している。
一方、 \sphinxcode{\sphinxupquote{ix0}} , \sphinxcode{\sphinxupquote{jx0}} , \sphinxcode{\sphinxupquote{kx0}} はそれぞれx, y, z方向のMPIプロセスの数を定義している。全MPIプロセス数は \sphinxcode{\sphinxupquote{ix0*jx0*kx0}} となり、各方向の全体の格子点数はそれぞれ \sphinxcode{\sphinxupquote{nx*ix0}} , \sphinxcode{\sphinxupquote{ny*jx0}} , \sphinxcode{\sphinxupquote{nz*kx0}} となる。


\section{初期条件}
\label{\detokenize{start:id10}}
初期条件は、 \sphinxcode{\sphinxupquote{src/all/model.F90}} で設定している。基本的な光球計算などは、鉛直速度にランダムな値を入れて計算を始めている。


\section{追加条件}
\label{\detokenize{start:id11}}
ある程度計算を行った後に、続きの計算として少し設定を変えたい場合の手続きを示す。
例えば、磁場なしの熱対流計算を行った後に、磁場を加える場合などに有効である。

\sphinxcode{\sphinxupquote{run/d001}} での計算を:code:\sphinxtitleref{run/d002} に移す場合について説明する。
\sphinxcode{\sphinxupquote{run/d001}} の下に \sphinxcode{\sphinxupquote{copy\_caseid.py}} というスクリプトがあるのでそれを実行する(なければ堀田からもらう)

\begin{sphinxVerbatim}[commandchars=\\\{\}]
\PYG{n}{python} \PYG{n}{copy\PYGZus{}caseid}\PYG{o}{.}\PYG{n}{py}
\end{sphinxVerbatim}

実行すると

\begin{sphinxVerbatim}[commandchars=\\\{\}]
\PYG{n}{Q1}\PYG{o}{.} \PYG{n}{Input} \PYG{n}{caseid} \PYG{k}{for} \PYG{n}{copy}\PYG{p}{,} \PYG{n}{like} \PYG{n}{d001}
\end{sphinxVerbatim}

と質問されるので、データを移す先のcaseidを \sphinxcode{\sphinxupquote{d002}} などと入力する。

次に

\begin{sphinxVerbatim}[commandchars=\\\{\}]
\PYG{n}{Q2}\PYG{o}{.} \PYG{n}{Input} \PYG{n}{time} \PYG{n}{step} \PYG{k}{for} \PYG{n}{copy}\PYG{p}{,} \PYG{n}{like} \PYG{l+m+mi}{10} \PYG{o+ow}{or} \PYG{n}{end}
\end{sphinxVerbatim}

と質問されるので、移動したいデータの時間ステップを \sphinxcode{\sphinxupquote{10}} などと入力する。
チェックポイントのデータはデフォルトでは、10回に一回しか出力していないので注意すること。

また、現在行った計算の最後の時間ステップのデータを移動したい時は \sphinxcode{\sphinxupquote{10}} などの代わりに
\sphinxcode{\sphinxupquote{end}} と入力する。
するとプログラム・データのコピーが始まる。すでに移動先(今回場合は \sphinxcode{\sphinxupquote{d002}} に \sphinxcode{\sphinxupquote{data}} ディレクトリがある場合は、コピーが始まらないので、削除してからコピーすること)。

また、コピーが終わると移動先の \sphinxcode{\sphinxupquote{data/cont\_log.txt}} に元データの情報が記載してある。


\section{スーパーコンピュータでのシェルスクリプト}
\label{\detokenize{start:id12}}
いくつかのスーパーコンピュータでジョブを投入するためのシェルスクリプトも \sphinxcode{\sphinxupquote{sh}} ディレクトリに用意している。
使用コア数などを変えたい時は、それぞれのスーパーコンピュータの使用説明書などを参照すること。
今後使うことのできるものだけをあげる。
\begin{itemize}
\item {} 
\sphinxcode{\sphinxupquote{fx.sh}} : 名大FX100

\item {} 
\sphinxcode{\sphinxupquote{ofp.sh}} : Oakforest\sphinxhyphen{}PACS

\item {} 
\sphinxcode{\sphinxupquote{xc.sh}} : CfCA XC50

\end{itemize}


\section{初期条件データを受け取った場合}
\label{\detokenize{start:id13}}
熱対流が統計的定常に達するまでは非常に時間がかかるために、この計算が非常に困難になる。そのため、堀田がデータを提供することがある。堀田は \sphinxcode{\sphinxupquote{data}} ディレクトリを丸ごと提供する。

このディレクトリに \sphinxcode{\sphinxupquote{cont\_log.txt}} というファイルがあるので、そこに示されている計算設定の情報を見て同じようになるように計算を設定する。

このディレクトリを書く実行のディレクトリの配下におき、実行すると続きの計算が始まる。


\chapter{R2D2を使うための環境設定}
\label{\detokenize{environment:r2d2}}\label{\detokenize{environment::doc}}
R2D2コードを使って計算するだけならば任意のfortranコンパイラ, FFTW, MPIのみがあれば良い。
Pythonコードを使って解析する場合は、いくつかのモジュールが必要なので、そのインストールの方法もここで説明する。


\section{Fortranコードの環境設定}
\label{\detokenize{environment:fortran}}

\subsection{Mac}
\label{\detokenize{environment:mac}}
Homebrewを用いて、必要なコンパイラ・ライブラリをインストールすることを推奨している。
コンパイラとFFTWのインクルードファイルとライブラリの位置だけ指定すれば良いので、
任意の方法でインストールして構わない。Homebrew以外を用いる場合は、便宜make/Makefileを編集すること。

Homebrewのインストール

\begin{sphinxVerbatim}[commandchars=\\\{\}]
\PYG{o}{/}\PYG{n}{usr}\PYG{o}{/}\PYG{n+nb}{bin}\PYG{o}{/}\PYG{n}{ruby} \PYG{o}{\PYGZhy{}}\PYG{n}{e} \PYG{l+s+s2}{\PYGZdq{}}\PYG{l+s+s2}{\PYGZdl{}(curl \PYGZhy{}fsSL https://raw.githubusercontent.com/Homebrew/install/master/install)}\PYG{l+s+s2}{\PYGZdq{}}
\end{sphinxVerbatim}

gfortranのインストール

\begin{sphinxVerbatim}[commandchars=\\\{\}]
\PYG{n}{brew} \PYG{n}{install} \PYG{n}{gcc}
\end{sphinxVerbatim}

OpenMPIのインストール

\begin{sphinxVerbatim}[commandchars=\\\{\}]
\PYG{n}{brew} \PYG{n}{install} \PYG{n}{openmpi}
\end{sphinxVerbatim}

FFTWのインストール

\begin{sphinxVerbatim}[commandchars=\\\{\}]
\PYG{n}{brew} \PYG{n}{install} \PYG{n}{fftw}
\end{sphinxVerbatim}


\subsection{Linux (Ubuntu 18.04)}
\label{\detokenize{environment:linux-ubuntu-18-04}}
ここでは、Ubuntu 18.04の場合のみを説明する。

gfortranのインストール

\begin{sphinxVerbatim}[commandchars=\\\{\}]
\PYG{n}{sudo} \PYG{n}{apt}\PYG{o}{\PYGZhy{}}\PYG{n}{get} \PYG{n}{install} \PYG{n}{gfortran}
\end{sphinxVerbatim}

OpenMPIのインストール

\begin{sphinxVerbatim}[commandchars=\\\{\}]
\PYG{n}{sudo} \PYG{n}{apt}\PYG{o}{\PYGZhy{}}\PYG{n}{get} \PYG{n}{install} \PYG{n}{openmpi}\PYG{o}{\PYGZhy{}}\PYG{n}{doc} \PYG{n}{openmpi}\PYG{o}{\PYGZhy{}}\PYG{n+nb}{bin} \PYG{n}{libopenmpi}\PYG{o}{\PYGZhy{}}\PYG{n}{dev}
\end{sphinxVerbatim}

FFTWのインストール

\begin{sphinxVerbatim}[commandchars=\\\{\}]
\PYG{n}{sudo} \PYG{n}{apt}\PYG{o}{\PYGZhy{}}\PYG{n}{get} \PYG{n}{install} \PYG{n}{libfftw3}\PYG{o}{\PYGZhy{}}\PYG{n}{dev}
\end{sphinxVerbatim}


\section{Pythonコードの環境設定}
\label{\detokenize{environment:python}}
Anacondaをインストールし、以下に示すモジュール群をインストールする。
MacとLinuxで共通する部分が多いのでまとめて説明を記す。

\sphinxhref{https://www.anaconda.com/}{Anacondaのウェブサイト} から対応するインストーラーをダウンロードする。
\begin{itemize}
\item {} \begin{description}
\item[{Mac}] \leavevmode
dmgファイルをダウンロードして、インストール。インストールされるPATHが変わることが多いが、探してPATHを通す。 \sphinxcode{\sphinxupquote{/anaconda/bin}} や \sphinxcode{\sphinxupquote{\textasciitilde{}/opt/anaconda/bin}} など

\end{description}

\item {} \begin{description}
\item[{Linux}] \leavevmode
ダウンロードしてきたシェルスクリプトファイルのあるディレクトリで
.. code:

\begin{sphinxVerbatim}[commandchars=\\\{\}]
\PYG{n}{bash} \PYG{o}{\PYGZti{}}\PYG{o}{/}\PYG{n}{Anaconda}\PYG{o}{*}\PYG{o}{*}\PYG{o}{*}\PYG{o}{.}\PYG{n}{sh}
\end{sphinxVerbatim}

インストールするディレクトリは \sphinxcode{\sphinxupquote{/ホームディレクトリ/anaconda3}} とする。
\sphinxcode{\sphinxupquote{/ホームディレクトリ/anaconda3}} にPATHを通す。
スパコンのログインノードなどでもインストール方法は共通である。

\end{description}

\end{itemize}


\subsection{ipythonの初期設定}
\label{\detokenize{environment:ipython}}
以下は必須ではないが、ipythonを使う時の初期設定ファイルである。
\sphinxcode{\sphinxupquote{\textasciitilde{}/.ipython/profile\_default/startup/00\_first.py}}
というファイルを作りそこに以下のように記す。

\begin{sphinxVerbatim}[commandchars=\\\{\}]
\PYG{k+kn}{import} \PYG{n+nn}{sys}\PYG{o}{,} \PYG{n+nn}{os}
\PYG{k+kn}{import} \PYG{n+nn}{matplotlib}\PYG{n+nn}{.}\PYG{n+nn}{pyplot} \PYG{k}{as} \PYG{n+nn}{plt}
\PYG{k+kn}{import} \PYG{n+nn}{scipy} \PYG{k}{as} \PYG{n+nn}{sp}
\PYG{k+kn}{import} \PYG{n+nn}{numpy} \PYG{k}{as} \PYG{n+nn}{np}
\PYG{k+kn}{from} \PYG{n+nn}{matplotlib}\PYG{n+nn}{.}\PYG{n+nn}{pyplot} \PYG{k+kn}{import} \PYG{n}{pcolormesh}\PYG{p}{,}\PYG{n}{plot}\PYG{p}{,}\PYG{n}{clf}\PYG{p}{,}\PYG{n}{close}
\PYG{k+kn}{from} \PYG{n+nn}{numpy} \PYG{k+kn}{import} \PYG{n}{sin}\PYG{p}{,}\PYG{n}{cos}\PYG{p}{,}\PYG{n}{tan}\PYG{p}{,}\PYG{n}{arcsin}\PYG{p}{,}\PYG{n}{arccos}\PYG{p}{,}\PYG{n}{arctan}\PYG{p}{,}\PYG{n}{exp}\PYG{p}{,}\PYG{n}{log}\PYG{p}{,}\PYG{n}{log2}\PYG{p}{,}\PYG{n}{log10}\PYG{p}{,}\PYG{n}{mod}\PYG{p}{,}\PYG{n}{sqrt}\PYG{p}{,}\PYG{n}{absolute}\PYG{p}{,}\PYG{n}{sinh}\PYG{p}{,}\PYG{n}{cosh}\PYG{p}{,}\PYG{n}{tanh}\PYG{p}{,}\PYG{n}{pi}\PYG{p}{,}\PYG{n}{arange}
\PYG{n}{plt}\PYG{o}{.}\PYG{n}{ion}\PYG{p}{(}\PYG{p}{)}
\PYG{k+kn}{from} \PYG{n+nn}{IPython}\PYG{n+nn}{.}\PYG{n+nn}{core}\PYG{n+nn}{.}\PYG{n+nn}{magic} \PYG{k+kn}{import} \PYG{n}{register\PYGZus{}line\PYGZus{}magic}
\PYG{n+nd}{@register\PYGZus{}line\PYGZus{}magic}
\PYG{k}{def} \PYG{n+nf}{r}\PYG{p}{(}\PYG{n}{line}\PYG{p}{)}\PYG{p}{:}
    \PYG{n}{get\PYGZus{}ipython}\PYG{p}{(}\PYG{p}{)}\PYG{o}{.}\PYG{n}{run\PYGZus{}line\PYGZus{}magic}\PYG{p}{(}\PYG{l+s+s1}{\PYGZsq{}}\PYG{l+s+s1}{run}\PYG{l+s+s1}{\PYGZsq{}}\PYG{p}{,} \PYG{l+s+s1}{\PYGZsq{}}\PYG{l+s+s1}{ \PYGZhy{}i }\PYG{l+s+s1}{\PYGZsq{}} \PYG{o}{+} \PYG{n}{line}\PYG{p}{)}
\PYG{k}{del} \PYG{n}{r}
\end{sphinxVerbatim}

最後に記した設定によって、

\begin{sphinxVerbatim}[commandchars=\\\{\}]
\PYG{n}{r} \PYG{p}{(}\PYG{n}{Pythonスクリプト名}\PYG{p}{)}
\end{sphinxVerbatim}

とするだけで、スクリプトを実行できるようになる。


\subsection{Googleスプレッドシート利用}
\label{\detokenize{environment:google}}
計算設定などをGoogleスプレッドシートにまとめておくと便利である。
R2D2では、Pythonから直接Googleスプレッドシートに送信する方法を提供しているので、利用したい方は検討されたい。

手順については、 \sphinxhref{https://qiita.com/akabei/items/0eac37cb852ad476c6b9}{こちら} を参考にしたが、少し手順が違うのでこのページでも解説する。

まずは関連するモジュールのインストール。

\begin{sphinxVerbatim}[commandchars=\\\{\}]
pip install gspread
pip install oauth2client
\end{sphinxVerbatim}

プロキシなどの影響でpipが使えない時は以下のようにする

gspreadのインストール

\begin{sphinxVerbatim}[commandchars=\\\{\}]
git clone git@github.com:burnash/gspread.git
\PYG{n+nb}{cd} gspread
ipython setup.py install
\end{sphinxVerbatim}

oauth2clientのインストール

\begin{sphinxVerbatim}[commandchars=\\\{\}]
git clone git@github.com:googleapis/oauth2client.git
\PYG{n+nb}{cd} oauth2client
ipython setup.py install
\end{sphinxVerbatim}


\subsubsection{プロジェクト作成}
\label{\detokenize{environment:id2}}
ウェブブラウザで \sphinxurl{https://console.developers.google.com/cloud-resource-manager?pli=1} にアクセス。

\noindent\sphinxincludegraphics[width=350\sphinxpxdimen]{{gen_project1}.png}

「プロジェクトを作成」として、プロジェクトを作成

\noindent\sphinxincludegraphics[width=400\sphinxpxdimen]{{gen_project2}.png}

プロジェクト名はR2D2, 場所は組織なしとする。


\subsubsection{API有効化}
\label{\detokenize{environment:api}}
\noindent\sphinxincludegraphics[width=400\sphinxpxdimen]{{google_drive1}.png}

次に検索窓にGoogle Driveと打ち込んで、Google DriveのAPIを検索

\noindent\sphinxincludegraphics[width=400\sphinxpxdimen]{{google_drive2}.png}

Google Drive APIを有効にする。

\noindent\sphinxincludegraphics[width=400\sphinxpxdimen]{{google_sheet1}.png}

同様にGoogle Sheetsと検索

\noindent\sphinxincludegraphics[width=400\sphinxpxdimen]{{google_sheet2}.png}

Google Sheets APIを有効化


\subsubsection{サービスアカウント作成}
\label{\detokenize{environment:id3}}
\noindent\sphinxincludegraphics[width=400\sphinxpxdimen]{{service_account1}.png}

Google APIロゴ \(\rightarrow\) 認証情報 \(\rightarrow\) サービスアカウントとたどる。

\noindent\sphinxincludegraphics[width=400\sphinxpxdimen]{{service_account2}.png}

サービスアカウント名はR2D2とする

\noindent\sphinxincludegraphics[width=400\sphinxpxdimen]{{service_account3}.png}

役割は編集者を選択

\noindent\sphinxincludegraphics[width=400\sphinxpxdimen]{{service_account4}.png}

キーの生成ではJSONを選択し、キーを生成する。
ダウンロードしたファイルは、使用する計算機のホームディレクトリにjsonというディレクトリを作成し、その下に配置する。そのディレクトリには、このjsonファイル以外には何も置かないこと。


\subsubsection{スプレッドシート作成}
\label{\detokenize{environment:id4}}
以下のウェブサイトからGoogleスプレッドシートを作成
\sphinxurl{https://docs.google.com/spreadsheets/create}

名前はプロジェクト名とする。R2D2では、pyディレクトリの上のディレクトリ名を読みそれをスプレッドシートの名前として情報を送るので、ディレクトリと同じ名前にする。

\noindent\sphinxincludegraphics[width=400\sphinxpxdimen]{{spread_sheet1}.png}

講習会ではR2D2としておく。

\noindent\sphinxincludegraphics[width=400\sphinxpxdimen]{{spread_sheet2}.png}

共有をクリックし、ダウンロードしたjsonファイルの中のclient\_email行のEメールアドレスをコピーして、貼り付け。ここまでで、R2D2からGoogleスプレッドシートにアクセスできるようになる。


\section{IDLコードの環境設定}
\label{\detokenize{environment:idl}}
システムにIDLをインストールすれば、それのみで使える。ここでは説明しない。


\chapter{典型的計算例}
\label{\detokenize{typical_case:id1}}\label{\detokenize{typical_case::doc}}
ここでは、典型的計算設定について紹介する。


\section{小規模局所光球計算}
\label{\detokenize{typical_case:id2}}
Vögler et al., 2005などで


\chapter{方程式}
\label{\detokenize{equation:id1}}\label{\detokenize{equation::doc}}
R2D2で解く方程式は以下である。現状では、デカルト座標 \((x,y,z)\)  のみを提供している。数値計算コードの中では、\(x\) を重力方向(鉛直方向)に取っているが、論文を書く際は各自適切に判断されたい。


\section{磁気流体力学}
\label{\detokenize{equation:id2}}
磁気流体力学の方程式は以下を解いている。
\begin{equation*}
\begin{split}\frac{\partial \rho_1}{\partial t} &= - \frac{1}{\xi^2}\nabla\cdot
\left(\rho \boldsymbol{v}\right) \\
\frac{\partial}{\partial t}\left(\rho \boldsymbol{v}\right) &=
-\nabla\cdot\left(\rho\boldsymbol{vv}\right)
- \nabla p_1 - \rho_1 g\boldsymbol{e}_x
+\frac{1}{4\pi}\left(\nabla\times\boldsymbol{B}\right)
\times\boldsymbol{B} \\
\frac{\partial \boldsymbol{B}}{\partial t} &=
\nabla\times\left(\boldsymbol{v\times B}\right)
\\
\rho T \frac{\partial s_1}{\partial t} &= -\rho T
\left(\boldsymbol{v}\cdot\nabla\right) s + Q_\mathrm{rad} \\
p_1 &= p_1(\rho_1,s_1,x)\end{split}
\end{equation*}
ここで \(\rho\) は密度、\(\boldsymbol{v}\) は流体速度、\(\boldsymbol{B}\) は磁場、 \(s\) はエントロピー、\(p\) は圧力、 \(T\) は温度、 \(g\) は重力加速度、 \(Q_\mathrm{rad}\) は輻射による加熱率である。

R2D2では熱力学量を以下のように時間的に一定で \(x\) 方向の依存性のみを持つ0次の量とそこから擾乱の1次の量に分けている。
\begin{equation*}
\begin{split}\rho &= \rho_0 + \rho_1 \\
p &= p_0 + p_1 \\
s &= s_0 + s_1 \\
T &= T_0 + T_1 \\\end{split}
\end{equation*}
太陽内部では、\(\rho_1 << \rho_0\) などが成り立っているが、太陽表面では熱対流による擾乱と背景場は同程度となるので、R2D2の中では \(\rho_1 << \rho_0\) などは仮定しない。0次の量はModel Sを参考にして計算をしている。詳細は出版予定の論文Hotta \& Iijima, in prep (2020?)を参照されたい。


\section{輻射輸送}
\label{\detokenize{equation:id3}}
最終更新日:2020年04月28日


\chapter{コード構造}
\label{\detokenize{code:id1}}\label{\detokenize{code::doc}}

\chapter{数値スキーム}
\label{\detokenize{scheme:id1}}\label{\detokenize{scheme::doc}}

\section{MHDスキーム}
\label{\detokenize{scheme:mhd}}

\subsection{空間微分}
\label{\detokenize{scheme:id2}}
R2D2では、4次の中央差分を用いている。格子間隔が一様な場合には中央差分では微分は
\begin{equation*}
\begin{split}\left(\frac{\partial q}{\partial x}\right)_i =\frac{-q_{i+2}+8q_{i+1}-8q_{i-1}+q_{i-2}}{12\Delta x_i}\end{split}
\end{equation*}
となる。R2D2では、非一様な格子間隔にも対応しており、


\subsection{時間積分}
\label{\detokenize{scheme:id3}}
R2D2では、

最終更新日:2020年04月28日


\chapter{パラメータ}
\label{\detokenize{parameter:id1}}\label{\detokenize{parameter::doc}}
最終更新日:2020年04月28日


\chapter{座標生成}
\label{\detokenize{geometry:id1}}\label{\detokenize{geometry::doc}}
R2D2では中央差分法を用いているが、そのほとんどは数値フラックスを用いて書き直すことでき、提供される \sphinxcode{\sphinxupquote{x}} , \sphinxcode{\sphinxupquote{y}} , \sphinxcode{\sphinxupquote{z}} などは \sphinxstylestrong{セル中心} で定義される。よって計算領域内の最初のグリッドは、計算境界から半グリッド進んだところにある。

また、R2D2では一様グリッドと非一様グリッドどちらでも計算できるようにしている。


\section{一様グリッド}
\label{\detokenize{geometry:id2}}
一様グリッドを用いるときは
\begin{itemize}
\item {} 
格子間隔を計算する

\end{itemize}
\begin{equation*}
\begin{split}\Delta x = \frac{x_\mathrm{max} - x_\mathrm{min}}{N_x}\end{split}
\end{equation*}
ここで、コードでは、配列の要素数には \sphinxcode{\sphinxupquote{margin}} も含むので
\(N_x\) を計算するには \sphinxcode{\sphinxupquote{margin}} の部分を引く必要があることに注意。
\begin{itemize}
\item {} 
\(x_1\) を設定。\sphinxcode{\sphinxupquote{margin}} の分も考慮して計算する。

\item {} 
\sphinxcode{\sphinxupquote{do loop}} で順次足していく

\end{itemize}

コードは以下のようになる

\begin{sphinxVerbatim}[commandchars=\\\{\}]
\PYG{n}{dx\PYGZus{}unif} \PYG{o}{=} \PYG{p}{(}\PYG{n}{xmax}\PYG{o}{\PYGZhy{}}\PYG{n}{xmin}\PYG{p}{)}\PYG{o}{/}\PYG{k+kt}{real}\PYG{p}{(}\PYG{n}{ix00}\PYG{o}{\PYGZhy{}}\PYG{l+m+mi}{2}\PYG{o}{*}\PYG{n}{marginx}\PYG{p}{)}
\PYG{n}{x00}\PYG{p}{(}\PYG{l+m+mi}{1}\PYG{p}{)} \PYG{o}{=} \PYG{n}{xmin} \PYG{o}{+} \PYG{p}{(}\PYG{l+m+mf}{0.5d0}\PYG{o}{\PYGZhy{}}\PYG{n+nb}{dble}\PYG{p}{(}\PYG{n}{marginx}\PYG{p}{)}\PYG{p}{)}\PYG{o}{*}\PYG{n}{dx\PYGZus{}unif}
\PYG{k}{if}\PYG{p}{(}\PYG{n}{xdcheck} \PYG{o}{==} \PYG{l+m+mi}{2}\PYG{p}{)} \PYG{k}{then}
\PYG{k}{    }\PYG{k}{do }\PYG{n}{i} \PYG{o}{=} \PYG{l+m+mi}{1}\PYG{o}{+}\PYG{n}{i1}\PYG{p}{,}\PYG{n}{ix00}
        \PYG{n}{x00}\PYG{p}{(}\PYG{n}{i}\PYG{p}{)} \PYG{o}{=} \PYG{n}{x00}\PYG{p}{(}\PYG{n}{i}\PYG{o}{\PYGZhy{}}\PYG{n}{i1}\PYG{p}{)} \PYG{o}{+} \PYG{n}{dx\PYGZus{}unif}
    \PYG{n}{enddo}
\PYG{n}{endif}
\end{sphinxVerbatim}


\section{非一様グリッド}
\label{\detokenize{geometry:id3}}
非一様グリッドを用いるときは、太陽光球付近は、輻射輸送のために一様なグリッド、ある程度の深さから格子間隔が線形に増加する非一様グリッドを使うことにしている。光球近くは、光球をちゃんと解像するために一様グリッド、光球からある程度進むと、非一様グリッドを採用することにしている。
実際の構造は以下のようになっている。非一様グリッド領域の両端2つのグリッド間隔は一様グリッドをとるようにしている。

fortranのコードの中では
\begin{itemize}
\item {} 
\(\Delta x_0\) \(\rightarrow\) \sphinxcode{\sphinxupquote{dx00}} : 一様グリッドでの格子点間隔

\item {} 
\(i_\mathrm{x\left(uni\right)}\) \(\rightarrow\) \sphinxcode{\sphinxupquote{ix\_ununi}}: 一様グリッドの格子点数

\item {} 
\(x_\mathrm{ran}\) \(\rightarrow\) \sphinxcode{\sphinxupquote{xrange}}: 領域サイズ

\item {} 
\(x_\mathrm{ran0}\) \(\rightarrow\) \sphinxcode{\sphinxupquote{xrange0}}: 一様グリッドの領域サイズ

\item {} 
\(x_\mathrm{ran1}\) \(\rightarrow\) \sphinxcode{\sphinxupquote{xrange1}}: 非一様グリッドの領域サイズ

\item {} 
\(n_x\) \(\rightarrow\) \sphinxcode{\sphinxupquote{nxx}} : 非一様グリッドの格子点数

\end{itemize}

\noindent\sphinxincludegraphics[width=700\sphinxpxdimen]{{ununiform_grid}.png}

一様グリッドでのグリッド間隔は \(\Delta x_0\) として、非一様グリッドでは \(\delta x\) ずつグリッド間隔が大きくなっていくとする。
\begin{equation*}
\begin{split}x_\mathrm{tran}&={\color{red} \frac{1}{2} \Delta x_0} + {\color{blue} \Delta x_0} + \Delta x_0
+ \left(\Delta x_0 + \delta x\right)
+ \left(\Delta x_0 + 2\delta x\right) + [...] \\
&+ \left[\Delta x_0 + \left(n_x - 4\right)\delta x\right]
+ {\color{blue}\left[\Delta x_0 + \left(n_x - 4\right)\delta x\right]}
+ {\color{red}\frac{1}{2}\left[\Delta x_0 + \left(n_x - 4\right)\delta x\right]} \\
&= {\color{red} \Delta x_0 + \frac{1}{2}\left(n_x-4\right)\delta x}
+{\color{blue} 2\Delta x_0 + \left(n_x - 4\right)\delta x}
+ \sum_{n=0}^{n_x - 4}\left(\Delta x_0 + n\delta x\right) \\
&= 3\Delta x_0 + \frac{3\left(n_x-4\right)\delta x}{2}
+ \frac{\left[2\Delta x_0 + \left(n_x-4\right)\delta x\right]\left(n_x - 3\right)}{2} \\
&= n_x \Delta x_0 + \frac{1}{2} n_x\left(n_x - 4\right)\delta x\end{split}
\end{equation*}
この関係式より、グリッド間隔の増分 \(\delta x\) を以下のように求めることができる。
\begin{equation*}
\begin{split}\delta x = \frac{2\left(x_\mathrm{tran} - n_x\Delta x_0\right)}{\left(n_x - 4\right)n_x}\end{split}
\end{equation*}

\chapter{境界条件}
\label{\detokenize{boundary:id1}}\label{\detokenize{boundary::doc}}
論文を書くときは
\begin{itemize}
\item {} 
\(x\) , \(y\) : 水平方向

\item {} 
\(z\) : 鉛直方向

\end{itemize}

となっているが、R2D2のコード内では
\begin{itemize}
\item {} 
\sphinxcode{\sphinxupquote{x}}: 鉛直方向

\item {} 
\sphinxcode{\sphinxupquote{y}}, \sphinxcode{\sphinxupquote{z}}: 水平方向

\end{itemize}

となっている。この取扱説明書では、コードに合わせた表記を用いる。

また、対称・反対称とは以下のような状況を表す。

\noindent\sphinxincludegraphics[width=450\sphinxpxdimen]{{bc_sym}.png}


\section{上部境界}
\label{\detokenize{boundary:id2}}

\subsection{ポテンシャル磁場}
\label{\detokenize{boundary:id3}}
磁場があるときは、上部ではポテンシャル磁場境界条件を使う。


\section{下部境界}
\label{\detokenize{boundary:id4}}
開く時
どの質量フラックスも対称にする。 計算領域内での質量を一定に保つために、水平に平均した密度
\(\langle \rho_1\rangle\) は反対称。 そこからのずれ
\(\rho_1 - \langle \rho_1 \rangle\) は対称な境界条件をとる。

一方、エントロピー \(s_1\) は上昇流で反対称、下降流で反対称な境界条件をとる。
この心は、開く境界条件を取るときは計算をしている領域の結果は信用するが、外から入ってくる物理量は、
計算領域に寄らないというものである。 下降流は現在計算している領域内部での情報を持って計算領域の外に出ていくので、
対称な境界条件を用いる。一方、上昇流は、計算している領域の外からの情報を持って計算領域に入ってくるので、
反対称な境界条件を用いて擾乱をゼロにする。これは元々のModel Sでの量を上昇流のエントロピーに用いるということである。

最終更新日:2020年04月28日


\chapter{人工粘性}
\label{\detokenize{artdif:id1}}\label{\detokenize{artdif::doc}}
最終更新日:2020年04月28日


\chapter{出力と読込}
\label{\detokenize{io:id1}}\label{\detokenize{io::doc}}

\section{出力}
\label{\detokenize{io:id2}}

\subsection{Fortranコード}
\label{\detokenize{io:fortran}}

\section{読込}
\label{\detokenize{io:id3}}
読み込みについては、PythonコードとIDLコードを用意しているが、開発の頻度が高いPythonコードの利用を推奨している。


\subsection{Pythonコード}
\label{\detokenize{io:module-R2D2}}\label{\detokenize{io:python}}\index{R2D2 (モジュール)@\spxentry{R2D2}\spxextra{モジュール}}
PythonでR2D2で定義された関数を使うには

\begin{sphinxVerbatim}[commandchars=\\\{\}]
\PYG{k+kn}{import} \PYG{n+nn}{R2D2}
\end{sphinxVerbatim}

として、モジュールを読み込む。R2D2には {\hyperref[\detokenize{io:R2D2.R2D2_data}]{\sphinxcrossref{\sphinxcode{\sphinxupquote{R2D2\_data}}}}} クラスが定義してあり、これをオブジェクト指向的に用いてデータを取り扱う。

以下にそれぞれの関数を示すが、docstringは記入してあるので

\begin{sphinxVerbatim}[commandchars=\\\{\}]
\PYG{n}{help}\PYG{p}{(}\PYG{n}{R2D2}\PYG{p}{)}
\PYG{n}{help}\PYG{p}{(}\PYG{n}{R2D2}\PYG{o}{.}\PYG{n}{R2D2\PYGZus{}data}\PYG{p}{)}
\end{sphinxVerbatim}

などとすると実行環境で、モジュール全体や各関数の簡単な説明を見ることができる。


\subsubsection{クラス}
\label{\detokenize{io:id4}}\index{R2D2\_data (R2D2 のクラス)@\spxentry{R2D2\_data}\spxextra{R2D2 のクラス}}

\begin{fulllineitems}
\phantomsection\label{\detokenize{io:R2D2.R2D2_data}}\pysiglinewithargsret{\sphinxbfcode{\sphinxupquote{class }}\sphinxcode{\sphinxupquote{R2D2.}}\sphinxbfcode{\sphinxupquote{R2D2\_data}}}{\emph{datadir}}{}
\end{fulllineitems}


データの読み込みには \sphinxcode{\sphinxupquote{R2D2}} モジュールの中で定義されている \sphinxcode{\sphinxupquote{R2D2\_data}} クラスを使う必要がある。

\begin{sphinxVerbatim}[commandchars=\\\{\}]
\PYG{k+kn}{import} \PYG{n+nn}{R2D2}
\PYG{n}{datadir} \PYG{o}{=} \PYG{l+s+s1}{\PYGZsq{}}\PYG{l+s+s1}{../run/d002/data}\PYG{l+s+s1}{\PYGZsq{}}
\PYG{n}{d} \PYG{o}{=} \PYG{n}{R2D2}\PYG{o}{.}\PYG{n}{R2D2\PYGZus{}data}\PYG{p}{(}\PYG{n}{datadir}\PYG{p}{)}
\end{sphinxVerbatim}

などとしてインスタンスを生成する。


\subsubsection{Attribute}
\label{\detokenize{io:attribute}}\index{p (R2D2.R2D2\_data の属性)@\spxentry{p}\spxextra{R2D2.R2D2\_data の属性}}

\begin{fulllineitems}
\phantomsection\label{\detokenize{io:R2D2.R2D2_data.p}}\pysigline{\sphinxcode{\sphinxupquote{R2D2\_data.}}\sphinxbfcode{\sphinxupquote{p}}}
基本的なパラメタ。格子点数 \sphinxcode{\sphinxupquote{ix}} や領域サイズ \sphinxcode{\sphinxupquote{xmax}} など。
インスタンス生成時に同時に読み込まれる。

\end{fulllineitems}

\index{qs (R2D2.R2D2\_data の属性)@\spxentry{qs}\spxextra{R2D2.R2D2\_data の属性}}

\begin{fulllineitems}
\phantomsection\label{\detokenize{io:R2D2.R2D2_data.qs}}\pysigline{\sphinxcode{\sphinxupquote{R2D2\_data.}}\sphinxbfcode{\sphinxupquote{qs}}}
ある高さの2次元のndarrayが含まれる辞書型。 {\hyperref[\detokenize{io:R2D2.R2D2_data.read_qq_select}]{\sphinxcrossref{\sphinxcode{\sphinxupquote{R2D2\_data.read\_qq\_select()}}}}} で読み込んだ結果。

\end{fulllineitems}

\index{qq (R2D2.R2D2\_data の属性)@\spxentry{qq}\spxextra{R2D2.R2D2\_data の属性}}

\begin{fulllineitems}
\phantomsection\label{\detokenize{io:R2D2.R2D2_data.qq}}\pysigline{\sphinxcode{\sphinxupquote{R2D2\_data.}}\sphinxbfcode{\sphinxupquote{qq}}}
3次元のnumpy array。計算領域全体のデータ。{\hyperref[\detokenize{io:R2D2.R2D2_data.read_qq}]{\sphinxcrossref{\sphinxcode{\sphinxupquote{R2D2\_data.read\_qq()}}}}} で読み込んだ結果。

\end{fulllineitems}

\index{qt (R2D2.R2D2\_data の属性)@\spxentry{qt}\spxextra{R2D2.R2D2\_data の属性}}

\begin{fulllineitems}
\phantomsection\label{\detokenize{io:R2D2.R2D2_data.qt}}\pysigline{\sphinxcode{\sphinxupquote{R2D2\_data.}}\sphinxbfcode{\sphinxupquote{qt}}}
2次元のnumpy array。ある光学的厚さの面でのデータ。現在は光学的厚さ1, 0.1, 0.01でのデータを出力している。 {\hyperref[\detokenize{io:R2D2.R2D2_data.read_qq_tau}]{\sphinxcrossref{\sphinxcode{\sphinxupquote{R2D2\_data.read\_qq\_tau()}}}}} で読み込んだ結果。

\end{fulllineitems}

\index{vc (R2D2.R2D2\_data の属性)@\spxentry{vc}\spxextra{R2D2.R2D2\_data の属性}}

\begin{fulllineitems}
\phantomsection\label{\detokenize{io:R2D2.R2D2_data.vc}}\pysigline{\sphinxcode{\sphinxupquote{R2D2\_data.}}\sphinxbfcode{\sphinxupquote{vc}}}
Fortranの計算の中で解析した結果。 {\hyperref[\detokenize{io:R2D2.R2D2_data.read_vc}]{\sphinxcrossref{\sphinxcode{\sphinxupquote{R2D2\_data.read\_vc()}}}}} で読み込んだ結果。

\end{fulllineitems}

\index{qc (R2D2.R2D2\_data の属性)@\spxentry{qc}\spxextra{R2D2.R2D2\_data の属性}}

\begin{fulllineitems}
\phantomsection\label{\detokenize{io:R2D2.R2D2_data.qc}}\pysigline{\sphinxcode{\sphinxupquote{R2D2\_data.}}\sphinxbfcode{\sphinxupquote{qc}}}
3次元のnumpy array。計算領域全体のデータ。Fortranの計算でチェックポイントのために出力しているデータを読み込む。主に解像度をあげたいときのために使う {\hyperref[\detokenize{io:R2D2.R2D2_data.read_qq_check}]{\sphinxcrossref{\sphinxcode{\sphinxupquote{R2D2\_data.read\_qq\_check()}}}}} で読み込んだ結果。

\end{fulllineitems}


{\hyperref[\detokenize{io:R2D2.R2D2_data.p}]{\sphinxcrossref{\sphinxcode{\sphinxupquote{R2D2\_data.p}}}}} については、\sphinxcode{\sphinxupquote{init.py}} などで

\begin{sphinxVerbatim}[commandchars=\\\{\}]
\PYG{k+kn}{import} \PYG{n+nn}{R2D2}
\PYG{n}{d} \PYG{o}{=} \PYG{n}{R2D2}\PYG{o}{.}\PYG{n}{R2D2\PYGZus{}data}\PYG{p}{(}\PYG{n}{datadir}\PYG{p}{)}
\PYG{k}{for} \PYG{n}{key} \PYG{o+ow}{in} \PYG{n}{R2D2}\PYG{o}{.}\PYG{n}{p}\PYG{p}{:}
    \PYG{n}{exec}\PYG{p}{(}\PYG{l+s+s1}{\PYGZsq{}}\PYG{l+s+si}{\PYGZpc{}s}\PYG{l+s+s1}{ = }\PYG{l+s+si}{\PYGZpc{}s}\PYG{l+s+si}{\PYGZpc{}s}\PYG{l+s+si}{\PYGZpc{}s}\PYG{l+s+s1}{\PYGZsq{}} \PYG{o}{\PYGZpc{}} \PYG{p}{(}\PYG{n}{key}\PYG{p}{,} \PYG{l+s+s1}{\PYGZsq{}}\PYG{l+s+s1}{R2D2.p[}\PYG{l+s+s1}{\PYGZdq{}}\PYG{l+s+s1}{\PYGZsq{}}\PYG{p}{,}\PYG{n}{key}\PYG{p}{,}\PYG{l+s+s1}{\PYGZsq{}}\PYG{l+s+s1}{\PYGZdq{}}\PYG{l+s+s1}{]}\PYG{l+s+s1}{\PYGZsq{}}\PYG{p}{)}\PYG{p}{)}
\end{sphinxVerbatim}

などとしているために、辞書型の \sphinxcode{\sphinxupquote{key}} を名前にする変数に値が代入されている。例えば、 \sphinxcode{\sphinxupquote{R2D2\_data.p{[}\textquotesingle{}ix\textquotesingle{}{]}}} と \sphinxcode{\sphinxupquote{ix}} には同じ値が入っている。


\subsubsection{Method}
\label{\detokenize{io:method}}
メソッドで指定する \sphinxcode{\sphinxupquote{datadir}} はデータの場所を示す変数。R2D2の計算を実行すると \sphinxcode{\sphinxupquote{data}} ディレクトリが生成されて、その中にデータが保存される。この場所を指定すれば良い。
\index{\_\_init\_\_() (R2D2.R2D2\_data のメソッド)@\spxentry{\_\_init\_\_()}\spxextra{R2D2.R2D2\_data のメソッド}}

\begin{fulllineitems}
\phantomsection\label{\detokenize{io:R2D2.R2D2_data.__init__}}\pysiglinewithargsret{\sphinxcode{\sphinxupquote{R2D2\_data.}}\sphinxbfcode{\sphinxupquote{\_\_init\_\_}}}{\emph{datadir}}{}
インスタンス生成時に実行されるメソッド。計算設定などのパラメタが読み込まれる。 {\hyperref[\detokenize{io:R2D2.R2D2_data.p}]{\sphinxcrossref{\sphinxcode{\sphinxupquote{R2D2\_data.p}}}}} にデータが保存される。
\begin{quote}\begin{description}
\item[{パラメータ}] \leavevmode
\sphinxstyleliteralstrong{\sphinxupquote{datadir}} (\sphinxstyleliteralemphasis{\sphinxupquote{str}}) \sphinxhyphen{}\sphinxhyphen{} データの場所

\end{description}\end{quote}

\end{fulllineitems}

\index{read\_qq\_select() (R2D2.R2D2\_data のメソッド)@\spxentry{read\_qq\_select()}\spxextra{R2D2.R2D2\_data のメソッド}}

\begin{fulllineitems}
\phantomsection\label{\detokenize{io:R2D2.R2D2_data.read_qq_select}}\pysiglinewithargsret{\sphinxcode{\sphinxupquote{R2D2\_data.}}\sphinxbfcode{\sphinxupquote{read\_qq\_select}}}{\emph{xs}, \emph{n}, \emph{silent}}{}
ある高さのデータのスライスを読み込む。戻り値を返さない時も {\hyperref[\detokenize{io:R2D2.R2D2_data.qs}]{\sphinxcrossref{\sphinxcode{\sphinxupquote{R2D2\_data.qs}}}}} にデータが保存される。
\begin{quote}\begin{description}
\item[{パラメータ}] \leavevmode\begin{itemize}
\item {} 
\sphinxstyleliteralstrong{\sphinxupquote{xs}} (\sphinxstyleliteralemphasis{\sphinxupquote{float}}) \sphinxhyphen{}\sphinxhyphen{} 読み込みたいデータの高さ

\item {} 
\sphinxstyleliteralstrong{\sphinxupquote{n}} (\sphinxstyleliteralemphasis{\sphinxupquote{int}}) \sphinxhyphen{}\sphinxhyphen{} 読み込みたい時間ステップ

\end{itemize}

\end{description}\end{quote}

\end{fulllineitems}

\index{read\_qq() (R2D2.R2D2\_data のメソッド)@\spxentry{read\_qq()}\spxextra{R2D2.R2D2\_data のメソッド}}

\begin{fulllineitems}
\phantomsection\label{\detokenize{io:R2D2.R2D2_data.read_qq}}\pysiglinewithargsret{\sphinxcode{\sphinxupquote{R2D2\_data.}}\sphinxbfcode{\sphinxupquote{read\_qq}}}{\emph{n}}{}
3次元のデータを読み込む。戻り値を返さない時も {\hyperref[\detokenize{io:R2D2.R2D2_data.qq}]{\sphinxcrossref{\sphinxcode{\sphinxupquote{R2D2\_data.qq}}}}} にデータが保存される。
\begin{quote}\begin{description}
\item[{パラメータ}] \leavevmode
\sphinxstyleliteralstrong{\sphinxupquote{n}} (\sphinxstyleliteralemphasis{\sphinxupquote{int}}) \sphinxhyphen{}\sphinxhyphen{} 読み込みたい時間ステップ

\end{description}\end{quote}

\end{fulllineitems}

\index{read\_qq\_tau() (R2D2.R2D2\_data のメソッド)@\spxentry{read\_qq\_tau()}\spxextra{R2D2.R2D2\_data のメソッド}}

\begin{fulllineitems}
\phantomsection\label{\detokenize{io:R2D2.R2D2_data.read_qq_tau}}\pysiglinewithargsret{\sphinxcode{\sphinxupquote{R2D2\_data.}}\sphinxbfcode{\sphinxupquote{read\_qq\_tau}}}{\emph{n}}{}
光学的厚さが一定の2次元のデータを読み込む。{\hyperref[\detokenize{io:R2D2.R2D2_data.qt}]{\sphinxcrossref{\sphinxcode{\sphinxupquote{R2D2\_data.qt}}}}} にデータが保存される。
\begin{quote}\begin{description}
\item[{パラメータ}] \leavevmode
\sphinxstyleliteralstrong{\sphinxupquote{n}} (\sphinxstyleliteralemphasis{\sphinxupquote{int}}) \sphinxhyphen{}\sphinxhyphen{} 読み込みたい時間ステップ

\end{description}\end{quote}

\end{fulllineitems}

\index{read\_time() (R2D2.R2D2\_data のメソッド)@\spxentry{read\_time()}\spxextra{R2D2.R2D2\_data のメソッド}}

\begin{fulllineitems}
\phantomsection\label{\detokenize{io:R2D2.R2D2_data.read_time}}\pysiglinewithargsret{\sphinxcode{\sphinxupquote{R2D2\_data.}}\sphinxbfcode{\sphinxupquote{read\_time}}}{\emph{n}}{}
時間を読み込む。
\begin{quote}\begin{description}
\item[{パラメータ}] \leavevmode
\sphinxstyleliteralstrong{\sphinxupquote{n}} (\sphinxstyleliteralemphasis{\sphinxupquote{int}}) \sphinxhyphen{}\sphinxhyphen{} 読み込みたい時間ステップ

\item[{戻り値}] \leavevmode
時間ステップでの時間

\end{description}\end{quote}

\end{fulllineitems}

\index{read\_vc() (R2D2.R2D2\_data のメソッド)@\spxentry{read\_vc()}\spxextra{R2D2.R2D2\_data のメソッド}}

\begin{fulllineitems}
\phantomsection\label{\detokenize{io:R2D2.R2D2_data.read_vc}}\pysiglinewithargsret{\sphinxcode{\sphinxupquote{R2D2\_data.}}\sphinxbfcode{\sphinxupquote{read\_vc}}}{\emph{n}}{}
Fortranコードの中で解析した計算結果を読み込む。戻り値を返さない時も {\hyperref[\detokenize{io:R2D2.R2D2_data.vc}]{\sphinxcrossref{\sphinxcode{\sphinxupquote{R2D2\_data.vc}}}}} にデータが保存される。
\begin{quote}\begin{description}
\item[{パラメータ}] \leavevmode
\sphinxstyleliteralstrong{\sphinxupquote{n}} (\sphinxstyleliteralemphasis{\sphinxupquote{int}}) \sphinxhyphen{}\sphinxhyphen{} 読み込みたい時間ステップ

\end{description}\end{quote}

\end{fulllineitems}

\index{read\_qq\_check() (R2D2.R2D2\_data のメソッド)@\spxentry{read\_qq\_check()}\spxextra{R2D2.R2D2\_data のメソッド}}

\begin{fulllineitems}
\phantomsection\label{\detokenize{io:R2D2.R2D2_data.read_qq_check}}\pysiglinewithargsret{\sphinxcode{\sphinxupquote{R2D2\_data.}}\sphinxbfcode{\sphinxupquote{read\_qq\_check}}}{\emph{n}}{}
チェックポイントのためのデータを読み込む {\hyperref[\detokenize{io:R2D2.R2D2_data.qc}]{\sphinxcrossref{\sphinxcode{\sphinxupquote{R2D2\_data.qc}}}}} にデータが保存される。
\begin{quote}\begin{description}
\item[{パラメータ}] \leavevmode
\sphinxstyleliteralstrong{\sphinxupquote{n}} (\sphinxstyleliteralemphasis{\sphinxupquote{int}}) \sphinxhyphen{}\sphinxhyphen{} 読み込みたい時間ステップ

\end{description}\end{quote}

\end{fulllineitems}



\subsection{IDLコード}
\label{\detokenize{io:idl}}
\sphinxhref{https://github.com/hottahd/R2D2\_idl}{GitHubの公開レポジトリ} に簡単な説明あり


\subsection{Paraview 使用}
\label{\detokenize{io:paraview}}
Paraviewを使用するためにデータ出力の関数が用意してある。
\index{R2D2.vtk.write\_3D() (R2D2 モジュール)@\spxentry{R2D2.vtk.write\_3D()}\spxextra{R2D2 モジュール}}

\begin{fulllineitems}
\phantomsection\label{\detokenize{io:R2D2.R2D2.vtk.write_3D}}\pysiglinewithargsret{\sphinxcode{\sphinxupquote{R2D2.vtk.}}\sphinxbfcode{\sphinxupquote{write\_3D}}}{}{}
\end{fulllineitems}



\section{バージョン履歴}
\label{\detokenize{io:id5}}\begin{itemize}
\item {} 
ver. 1.0: バージョン制を導入

\item {} 
ver. 1.1: 光学的厚さが0.1, 0.01の部分も出力することにした。qq\_in, vcをconfigのグローバル変数として取扱うことにした。

\item {} 
ver. 1.2: データ構造を変更。

\end{itemize}


\chapter{R2D2 pythonでのキーワードの説明}
\label{\detokenize{notation:r2d2-python}}\label{\detokenize{notation::doc}}
以下では、R2D2 pythonで使われている辞書型に含まれるキーの説明を行う
\begin{itemize}
\item {} 
キーの名前 (型) \sphinxhyphen{}\sphinxhyphen{} 説明 {[}単位{]}

\end{itemize}

というフォーマットを採用する。

R2D2では、\sphinxcode{\sphinxupquote{R2D2\_data}} というクラスを用意している。


\section{self.p {[}dictionary{]}}
\label{\detokenize{notation:self-p-dictionary}}
\begin{sphinxVerbatim}[commandchars=\\\{\}]
\PYG{k+kn}{import} \PYG{n+nn}{R2D2}
\PYG{n+nb+bp}{self} \PYG{o}{=} \PYG{n}{R2D2}\PYG{o}{.}\PYG{n}{R2D2\PYGZus{}data}\PYG{p}{(}\PYG{n}{datadir}\PYG{p}{)}
\end{sphinxVerbatim}

とすると、初期設定が読み込まれる。 \sphinxcode{\sphinxupquote{self}} は \sphinxcode{\sphinxupquote{R2D2\_data}} のオブジェクトであり、名前は任意である。 \sphinxcode{\sphinxupquote{init.py}} や \sphinxcode{\sphinxupquote{mov.py}} では、オブジェクト名は \sphinxcode{\sphinxupquote{d}} としてある。


\subsection{出力・時間に関する量}
\label{\detokenize{notation:id1}}\begin{itemize}
\item {} 
datadir (str) \sphinxhyphen{}\sphinxhyphen{} データの保存場所

\item {} 
nd (int) \sphinxhyphen{}\sphinxhyphen{} 現在までのアウトプット時間ステップ数(3次元データ)

\item {} 
nd\_tau (int) \sphinxhyphen{}\sphinxhyphen{} 現在までのアウトプット時間ステップ数(光学的厚さ一定のデータ)

\item {} 
dtout (float) \sphinxhyphen{}\sphinxhyphen{} 出力ケーデンス {[}s{]}

\item {} 
dtout\_tau (float) \sphinxhyphen{}\sphinxhyphen{} 光学的厚さ一定のデータの出力ケーデンス {[}s{]}

\item {} 
ifac (int) \sphinxhyphen{}\sphinxhyphen{} dtout/dtout\_tau

\item {} 
tend (float) \sphinxhyphen{}\sphinxhyphen{} 計算終了時間。大きく取ってあるためにこの時間まで計算することはあまりない {[}s{]}

\item {} 
swap (int) \sphinxhyphen{}\sphinxhyphen{} エンディアン指定。big endianは1、little endianは0。IDLの定義に従っている。

\item {} 
endian (char) \sphinxhyphen{}\sphinxhyphen{} エンディアン指定。big endianは \textgreater{} 、little endianは \textless{} 。pythonの定義に従っている。

\item {} 
m\_in (int) \sphinxhyphen{}\sphinxhyphen{} 光学的厚さ一定のデータを出力する変数の数

\item {} 
m\_tu (int) \sphinxhyphen{}\sphinxhyphen{} 光学的厚さ一定のデータの層の数

\end{itemize}


\subsection{座標に関する量}
\label{\detokenize{notation:id2}}\begin{itemize}
\item {} 
xdcheck (int) \sphinxhyphen{}\sphinxhyphen{} x軸方向に解いているか。解いていたら2、解いていなかったら1

\item {} 
ydcheck (int) \sphinxhyphen{}\sphinxhyphen{} y軸方向に解いているか。解いていたら2、解いていなかったら1

\item {} 
zdcheck (int) \sphinxhyphen{}\sphinxhyphen{} z軸方向に解いているか。解いていたら2、解いていなかったら1

\item {} 
margin (int) \sphinxhyphen{}\sphinxhyphen{} マージン(ゴーストセル)の数

\item {} 
nx (int) \sphinxhyphen{}\sphinxhyphen{} 1 MPIスレッドあたりのx方向の格子点の数

\item {} 
ny (int) \sphinxhyphen{}\sphinxhyphen{} 1 MPIスレッドあたりのy方向の格子点の数

\item {} 
nz (int) \sphinxhyphen{}\sphinxhyphen{} 1 MPIスレッドあたりのz方向の格子点の数

\item {} 
ix0 (int) \sphinxhyphen{}\sphinxhyphen{} x方向のMPI領域分割の数

\item {} 
jx0 (int) \sphinxhyphen{}\sphinxhyphen{} y方向のMPI領域分割の数

\item {} 
kx0 (int) \sphinxhyphen{}\sphinxhyphen{} z方向のMPI領域分割の数

\item {} 
ix (int) \sphinxhyphen{}\sphinxhyphen{} x方向の格子点数 ix0*nx

\item {} 
jx (int) \sphinxhyphen{}\sphinxhyphen{} y方向の格子点数 jx0*ny

\item {} 
kx (int) \sphinxhyphen{}\sphinxhyphen{} z方向の格子点数 kx0*nz

\item {} 
npe (int) \sphinxhyphen{}\sphinxhyphen{} 全MPIスレッドの数 \sphinxcode{\sphinxupquote{npe = ix0*jx0*kx0}}

\item {} 
mtype (int) \sphinxhyphen{}\sphinxhyphen{} 変数の数

\item {} 
xmax (float) \sphinxhyphen{}\sphinxhyphen{} x方向境界の位置(上限値) {[}cm{]}

\item {} 
xmin (float) \sphinxhyphen{}\sphinxhyphen{} x方向境界の位置(下限値) {[}cm{]}

\item {} 
ymax (float) \sphinxhyphen{}\sphinxhyphen{} y方向境界の位置(上限値) {[}cm{]}

\item {} 
ymin (float) \sphinxhyphen{}\sphinxhyphen{} y方向境界の位置(下限値) {[}cm{]}

\item {} 
zmax (float) \sphinxhyphen{}\sphinxhyphen{} z方向境界の位置(上限値) {[}cm{]}

\item {} 
zmin (float) \sphinxhyphen{}\sphinxhyphen{} z方向境界の位置(下限値) {[}cm{]}

\item {} 
x (float) {[}ix{]} \sphinxhyphen{}\sphinxhyphen{} x方向の座標 {[}cm{]}

\item {} 
y (float) {[}jx{]} \sphinxhyphen{}\sphinxhyphen{} y方向の座標 {[}cm{]}

\item {} 
z (float) {[}kx{]} \sphinxhyphen{}\sphinxhyphen{} z方向の座標 {[}cm{]}

\item {} 
xr (float) {[}ix{]} \sphinxhyphen{}\sphinxhyphen{} x/rsun

\item {} 
xn (float) {[}ix{]} \sphinxhyphen{}\sphinxhyphen{} \sphinxcode{\sphinxupquote{(x\sphinxhyphen{}rsun)*1.e\sphinxhyphen{}8}}

\item {} 
deep\_top\_flag (int) \sphinxhyphen{}\sphinxhyphen{}

\item {} 
ib\_excluded\_top (int) \sphinxhyphen{}\sphinxhyphen{}

\item {} 
rsun (float) {[}ix{]} \sphinxhyphen{}\sphinxhyphen{} 太陽半径 {[}cm{]}

\end{itemize}


\subsection{背景場に関する量}
\label{\detokenize{notation:id3}}\begin{itemize}
\item {} 
pr0 (float) {[}ix{]} \sphinxhyphen{}\sphinxhyphen{} 背景場の圧力 {[}dyn cm $^{\text{\sphinxhyphen{}2}}${]}

\item {} 
te0 (float) {[}ix{]} \sphinxhyphen{}\sphinxhyphen{} 背景場の温度 {[}K{]}

\item {} 
ro0 (float) {[}ix{]} \sphinxhyphen{}\sphinxhyphen{} 背景場の密度 {[}g cm $^{\text{\sphinxhyphen{}3}}${]}

\item {} 
se0 (float) {[}ix{]} \sphinxhyphen{}\sphinxhyphen{} 背景場のエントロピー {[}erg g $^{\text{\sphinxhyphen{}1}}$ K $^{\text{\sphinxhyphen{}1}}${]}

\item {} 
en0 (float) {[}ix{]} \sphinxhyphen{}\sphinxhyphen{} 背景場の内部エネルギー {[}erg cm $^{\text{\sphinxhyphen{}3}}${]}

\item {} 
op0 (float) {[}ix{]} \sphinxhyphen{}\sphinxhyphen{} 背景場のオパシティー {[}g $^{\text{\sphinxhyphen{}1}}$ cm $^{\text{\sphinxhyphen{}2}}${]}

\item {} 
tu0 (float) {[}ix{]} \sphinxhyphen{}\sphinxhyphen{} 背景場の光学的厚さ

\item {} 
dsedr0 (float) {[}ix{]} \sphinxhyphen{}\sphinxhyphen{} 背景場の鉛直エントロピー勾配 {[}erg g $^{\text{\sphinxhyphen{}1}}$ K $^{\text{\sphinxhyphen{}1}}$ cm $^{\text{\sphinxhyphen{}1}}${]}

\item {} 
dtedr0 (float) {[}ix{]} \sphinxhyphen{}\sphinxhyphen{} 背景場の鉛直温度勾配 {[}K cm $^{\text{\sphinxhyphen{}1}}${]}

\item {} 
dprdro (float) {[}ix{]} \sphinxhyphen{}\sphinxhyphen{} 背景場の \((\partial p/\partial \rho)_s\)

\item {} 
dprdse (float) {[}ix{]} \sphinxhyphen{}\sphinxhyphen{} 背景場の \((\partial p/\partial s)_\rho\)

\item {} 
dtedro (float) {[}ix{]} \sphinxhyphen{}\sphinxhyphen{} 背景場の \((\partial T/\partial \rho)_s\)

\item {} 
dtedse (float) {[}ix{]} \sphinxhyphen{}\sphinxhyphen{} 背景場の \((\partial T/\partial s)_\rho\)

\item {} 
dendro (float) {[}ix{]} \sphinxhyphen{}\sphinxhyphen{} 背景場の \((\partial e/\partial \rho)_s\)

\item {} 
dendse (float) {[}ix{]} \sphinxhyphen{}\sphinxhyphen{} 背景場の \((\partial e/\partial s)_\rho\)

\item {} 
gx (float) {[}ix{]} \sphinxhyphen{}\sphinxhyphen{} 重力加速度 {[}cm s $^{\text{\sphinxhyphen{}2}}${]}

\item {} 
kp (float) {[}ix{]} \sphinxhyphen{}\sphinxhyphen{} 放射拡散係数 {[}cm $^{\text{2}}$ s $^{\text{\sphinxhyphen{}1}}${]}

\item {} 
cp (float) {[}ix{]} \sphinxhyphen{}\sphinxhyphen{} 定圧比熱 {[}erg g $^{\text{\sphinxhyphen{}1}}$ K $^{\text{\sphinxhyphen{}1}}${]}

\item {} 
fa (float) {[}ix{]} \sphinxhyphen{}\sphinxhyphen{} 対流層の底付近の輻射によるエネルギーフラックス。光球付近では輻射輸送を直に解くために含まれないが、上部境界が光球にない場合は、上部境界付近の人工的なエネルギーフラックス(冷却が含まれる) {[}erg cm $^{\text{\sphinxhyphen{}2}}${]}

\item {} 
sa (float) {[}ix{]} \sphinxhyphen{}\sphinxhyphen{} 上記faによる加熱率 {[}erg cm $^{\text{\sphinxhyphen{}3}}${]}

\item {} 
xi (float) {[}ix{]} \sphinxhyphen{}\sphinxhyphen{} 音速抑制率

\item {} 
ix\_e (int) \sphinxhyphen{}\sphinxhyphen{} 状態方程式の密度の格子点数

\item {} 
jx\_e (int) \sphinxhyphen{}\sphinxhyphen{} 状態方程式のエントロピーの格子点数

\end{itemize}


\subsection{解析のためのデータ再配置(remap)に関する量}
\label{\detokenize{notation:remap}}\begin{itemize}
\item {} 
m2da (int) \sphinxhyphen{}\sphinxhyphen{} remapで出力した解析量の数

\item {} 
cl (char) {[}m2da{]} \sphinxhyphen{}\sphinxhyphen{} remapで出力した解析量の名前

\item {} 
jc (int) \sphinxhyphen{}\sphinxhyphen{} \sphinxcode{\sphinxupquote{self.vc{[}\textquotesingle{}vxp\textquotesingle{}{]}}} などで出力するスライスのy方向の位置

\item {} 
kc (int) \sphinxhyphen{}\sphinxhyphen{} 浮上磁場の中心と思っている場所を出力(あまり使わない)

\item {} 
ixr (int) \sphinxhyphen{}\sphinxhyphen{} remap後のx方向分割の数

\item {} 
jxr (int) \sphinxhyphen{}\sphinxhyphen{} remap後のy方向分割の数

\item {} 
iss (int) {[}npe{]} \sphinxhyphen{}\sphinxhyphen{}

\item {} 
iee (int) {[}npe{]} \sphinxhyphen{}\sphinxhyphen{}

\item {} 
jss (int) {[}npe{]} \sphinxhyphen{}\sphinxhyphen{}

\item {} 
jee (int) {[}npe{]} \sphinxhyphen{}\sphinxhyphen{}

\item {} 
iixl (int) {[}npe{]} \sphinxhyphen{}\sphinxhyphen{}

\item {} 
jjxl (int) {[}npe{]} \sphinxhyphen{}\sphinxhyphen{}

\item {} 
np\_ijr (int) {[}npe{]} \sphinxhyphen{}\sphinxhyphen{}

\item {} 
ir (int) {[}npe{]} \sphinxhyphen{}\sphinxhyphen{}

\item {} 
jr (int) {[}npe{]} \sphinxhyphen{}\sphinxhyphen{}

\item {} 
i2ir (int) {[}ix{]} \sphinxhyphen{}\sphinxhyphen{}

\item {} 
j2jr (int) {[}jx{]} \sphinxhyphen{}\sphinxhyphen{}

\end{itemize}


\section{self.qs {[}dictionary{]}}
\label{\detokenize{notation:self-qs-dictionary}}
\begin{sphinxVerbatim}[commandchars=\\\{\}]
\PYG{n}{xs} \PYG{o}{=} \PYG{l+m+mf}{0.99}\PYG{o}{*}\PYG{n}{rsun}
\PYG{n}{ns} \PYG{o}{=} \PYG{l+m+mi}{10}
\PYG{n+nb+bp}{self}\PYG{o}{.}\PYG{n}{read\PYGZus{}qq\PYGZus{}select}\PYG{p}{(}\PYG{n}{xs}\PYG{p}{,}\PYG{n}{ns}\PYG{p}{)}
\end{sphinxVerbatim}

として高さ \sphinxcode{\sphinxupquote{xs}} での二次元スライスを読み込む
\begin{itemize}
\item {} 
ro (float) {[}jx,kx{]} \sphinxhyphen{}\sphinxhyphen{} 密度の擾乱 \(\rho_1\) {[}g cm $^{\text{\sphinxhyphen{}3}}${]}

\item {} 
vx (float) {[}jx,kx{]} \sphinxhyphen{}\sphinxhyphen{} x方向の速度 \(v_x\) {[}cm s $^{\text{\sphinxhyphen{}1}}${]}

\item {} 
vy (float) {[}jx,kx{]} \sphinxhyphen{}\sphinxhyphen{} y方向の速度 \(v_y\) {[}cm s $^{\text{\sphinxhyphen{}1}}${]}

\item {} 
vz (float) {[}jx,kx{]} \sphinxhyphen{}\sphinxhyphen{} z方向の速度 \(v_z\) {[}cm s $^{\text{\sphinxhyphen{}1}}${]}

\item {} 
bx (float) {[}jx,kx{]} \sphinxhyphen{}\sphinxhyphen{} x方向の磁場 \(B_x\) {[}G{]}

\item {} 
by (float) {[}jx,kx{]} \sphinxhyphen{}\sphinxhyphen{} y方向の磁場 \(B_y\) {[}G{]}

\item {} 
bz (float) {[}jx,kx{]} \sphinxhyphen{}\sphinxhyphen{} z方向の磁場 \(B_z\) {[}G{]}

\item {} 
se (float) {[}jx,kx{]} \sphinxhyphen{}\sphinxhyphen{} エントロピーの擾乱 \(s_1\) {[}erg g $^{\text{\sphinxhyphen{}1}}$ K $^{\text{\sphinxhyphen{}1}}${]}

\item {} 
pr (float) {[}jx,kx{]} \sphinxhyphen{}\sphinxhyphen{} 圧力の擾乱 \(p_1\) {[}dyn cm $^{\text{\sphinxhyphen{}2}}${]}

\item {} 
te (float) {[}jx,kx{]} \sphinxhyphen{}\sphinxhyphen{} 温度の擾乱 \(T_1\) {[}K{]}

\item {} 
op (float) {[}jx,kx{]} \sphinxhyphen{}\sphinxhyphen{} 不透明度(オパシティー) \(\kappa\) {[}g $^{\text{\sphinxhyphen{}1}}$ cm $^{\text{\sphinxhyphen{}2}}${]}

\end{itemize}


\section{self.qq {[}dictionary{]}}
\label{\detokenize{notation:self-qq-dictionary}}
\sphinxcode{\sphinxupquote{self.qs}} と同様


\section{self.qt {[}dictionary{]}}
\label{\detokenize{notation:self-qt-dictionary}}
ほぼself.qsと同様だが、以下の追加量が保存してある。


\section{self.vc {[}dictionary{]}}
\label{\detokenize{notation:self-vc-dictionary}}

\chapter{Sphinx使用の覚書}
\label{\detokenize{sphinx:sphinx}}\label{\detokenize{sphinx::doc}}

\section{はじめに}
\label{\detokenize{sphinx:id1}}
Sphinxは、reStructuredTextからHTMLやLatexなどの
文章を生成するソフトウェアである。
\sphinxhref{https://www.sphinx-doc.org/ja/master/index.html}{Sphinxの公式サイト}
最近ではMarkdownでも記述できるが、結局最後のところはreStructuredTextで記述することになるので、現状では、Markdownは使用していない。このウェブサイトもSphinxで生成しているので、覚書をここに記す。


\section{インストール}
\label{\detokenize{sphinx:id3}}
ここではAnacondaがすでにインストールしてあるMac
にSphinxをインストールすることを考える。
基本的には以下のコマンドを実行するのみである。

\begin{sphinxVerbatim}[commandchars=\\\{\}]
pip install sphinx
\end{sphinxVerbatim}

Markdownを使いたい時は以下のようにする。

\begin{sphinxVerbatim}[commandchars=\\\{\}]
pip install commonmark recommonmark
\end{sphinxVerbatim}


\section{HTMLファイルの生成}
\label{\detokenize{sphinx:html}}
適当なディレクトリを作成(ここでは \sphinxcode{\sphinxupquote{test}} )とする。
そこで、 \sphinxcode{\sphinxupquote{sphinx\sphinxhyphen{}quickstart}} コマンドによりSphinxで作るドキュメントの
初期設定を行う。

\begin{sphinxVerbatim}[commandchars=\\\{\}]
mkdir \PYG{n+nb}{test} \PYG{c+c1}{\PYGZsh{} ディレクトリ作成}
\PYG{n+nb}{cd} \PYG{n+nb}{test}    \PYG{c+c1}{\PYGZsh{} ディレクトリに移動}
sphinx\PYGZhy{}quickstart
\end{sphinxVerbatim}

いくつか質問をされる。基本的には読めばわかる質問であるが
少し戸惑う質問を以下にあげる。
\begin{itemize}
\item {} 
プロジェクトのリリース: 1.0などとversionを答える。後に \sphinxcode{\sphinxupquote{conf.py}} を編集すれば変更可能

\item {} 
プロジェクトの言語: デフォルトは英語の \sphinxcode{\sphinxupquote{en}} であるが、日本語を使いたい時は \sphinxcode{\sphinxupquote{ja}} とする

\end{itemize}


\section{VS codeの利用}
\label{\detokenize{sphinx:vs-code}}
VS codeを利用すると快適にreStructuredTextを作成することができる。
\sphinxcode{\sphinxupquote{*.rst}} ファイルをVS codeで開くと自動で確認されるが、以下のプラグインをインストールする。

\noindent\sphinxincludegraphics[width=500\sphinxpxdimen]{{restructuredtext_vs}.png}

\sphinxcode{\sphinxupquote{Cmd+k Cmd+r}} で画面を分割してプレビューできる。正しい \sphinxcode{\sphinxupquote{conf.py}} の場所を設定する必要がある。


\section{環境設定}
\label{\detokenize{sphinx:id4}}
デフォルトの設定では、数式を書く時にMathjaxを使用するようで、数式の太字が意図するように表示されなかったのでsvgで出力することにした。
以下のように \sphinxcode{\sphinxupquote{conf.py}} に追記する。

\begin{sphinxVerbatim}[commandchars=\\\{\}]
\PYG{n}{extensions} \PYG{o}{+}\PYG{o}{=} \PYG{p}{[}\PYG{l+s+s1}{\PYGZsq{}}\PYG{l+s+s1}{sphinx.ext.imgmath}\PYG{l+s+s1}{\PYGZsq{}}\PYG{p}{]}
\PYG{n}{imgmath\PYGZus{}image\PYGZus{}format} \PYG{o}{=} \PYG{l+s+s1}{\PYGZsq{}}\PYG{l+s+s1}{svg}\PYG{l+s+s1}{\PYGZsq{}}
\PYG{n}{imgmath\PYGZus{}font\PYGZus{}size} \PYG{o}{=} \PYG{l+m+mi}{14}
\PYG{n}{pngmath\PYGZus{}latex}\PYG{o}{=}\PYG{l+s+s1}{\PYGZsq{}}\PYG{l+s+s1}{platex}\PYG{l+s+s1}{\PYGZsq{}}
\end{sphinxVerbatim}

また、ウェブサイトのテーマを変更することもできる。どのようなテーマがあるかは
\sphinxhref{https://sphinx-users.jp/cookbook/changetheme/index.html}{Sphinxのテーマ}
を参照。好きなテーマを選んで \sphinxcode{\sphinxupquote{conf.py}} に以下のように設定。

\begin{sphinxVerbatim}[commandchars=\\\{\}]
\PYG{n}{html\PYGZus{}theme} \PYG{o}{=} \PYG{l+s+s1}{\PYGZsq{}}\PYG{l+s+s1}{bizstyle}\PYG{l+s+s1}{\PYGZsq{}}
\PYG{n}{html\PYGZus{}theme\PYGZus{}options} \PYG{o}{=} \PYG{p}{\PYGZob{}}\PYG{l+s+s1}{\PYGZsq{}}\PYG{l+s+s1}{maincolor}\PYG{l+s+s1}{\PYGZsq{}} \PYG{p}{:} \PYG{l+s+s2}{\PYGZdq{}}\PYG{l+s+s2}{\PYGZsh{}696969}\PYG{l+s+s2}{\PYGZdq{}}\PYG{p}{\PYGZcb{}}
\end{sphinxVerbatim}

今後変更の余地あり。


\section{記法}
\label{\detokenize{sphinx:id6}}

\subsection{リンク}
\label{\detokenize{sphinx:id7}}\begin{itemize}
\item {} 
外部ウェブサイト

\end{itemize}

\begin{sphinxVerbatim}[commandchars=\\\{\}]
\PYG{l+s}{`Twitter }\PYG{l+s+si}{\PYGZlt{}https://twitter.com\PYGZgt{}}\PYG{l+s}{`\PYGZus{}}
\end{sphinxVerbatim}

などとすると
\begin{quote}

\sphinxhref{https://twitter.com}{Twitter}
\end{quote}

とリンクが生成される
\begin{itemize}
\item {} 
内部サイト

\end{itemize}

自分で作成しているドキュメントをリンクするには

\begin{sphinxVerbatim}[commandchars=\\\{\}]
\PYG{n+na}{:doc:}\PYG{n+nv}{`index`}
\end{sphinxVerbatim}

などとすると
\begin{quote}

{\hyperref[\detokenize{index::doc}]{\sphinxcrossref{\DUrole{doc}{R2D2マニュアル}}}}
\end{quote}

とリンクが生成される。


\subsection{コード}
\label{\detokenize{sphinx:id8}}
Sphinxでは、コードを直接記載することができる。また、言語に合わせてハイライトも可能。
コードの表記に選択できる言語は \sphinxhref{https://pygments.org/docs/lexers/}{Pygments} にまとめてある。

\begin{sphinxVerbatim}[commandchars=\\\{\}]
\PYG{p}{..} \PYG{o+ow}{code}\PYG{p}{::} \PYG{k}{fortran}

    \PYG{k}{implicit }\PYG{k}{none}
    \PYG{k+kt}{real}\PYG{p}{(}\PYG{n+nb}{KIND}\PYG{o}{=}\PYG{l+m+mf}{0.d0}\PYG{p}{)} \PYG{k+kd}{::} \PYG{n}{a}\PYG{p}{,}\PYG{n}{b}\PYG{p}{,}\PYG{n}{c}

    \PYG{n}{a} \PYG{o}{=} \PYG{l+m+mf}{1.d0}
    \PYG{n}{b} \PYG{o}{=} \PYG{l+m+mf}{2.d0}
    \PYG{n}{c} \PYG{o}{=} \PYG{n}{a} \PYG{o}{+} \PYG{n}{b}
\end{sphinxVerbatim}

このようにすると、以下のように表示される

\begin{sphinxVerbatim}[commandchars=\\\{\}]
\PYG{k}{implicit }\PYG{k}{none}
\PYG{k+kt}{real}\PYG{p}{(}\PYG{n+nb}{KIND}\PYG{o}{=}\PYG{l+m+mf}{0.d0}\PYG{p}{)} \PYG{k+kd}{::} \PYG{n}{a}\PYG{p}{,}\PYG{n}{b}\PYG{p}{,}\PYG{n}{c}

\PYG{n}{a} \PYG{o}{=} \PYG{l+m+mf}{1.d0}
\PYG{n}{b} \PYG{o}{=} \PYG{l+m+mf}{2.d0}
\PYG{n}{c} \PYG{o}{=} \PYG{n}{a} \PYG{o}{+} \PYG{n}{b}
\end{sphinxVerbatim}


\subsection{画像}
\label{\detokenize{sphinx:id9}}
画像の挿入には \sphinxcode{\sphinxupquote{image}} ディレクティブを使う。オプションで、画像サイズなどを調整できる。堀田はだいたいwidthで調整している。

\begin{sphinxVerbatim}[commandchars=\\\{\}]
\PYG{p}{..} \PYG{o+ow}{image}\PYG{p}{::} source/figs/R2D2\PYGZus{}logo.png
    \PYG{n+nc}{:width:} 350 px
\end{sphinxVerbatim}

とすると下記のように画像が挿入される。

\noindent\sphinxincludegraphics[width=350\sphinxpxdimen]{{R2D2_logo}.png}


\subsection{数式}
\label{\detokenize{sphinx:id10}}
SphinxではLatexを用いて数式を記述することができる。
1行の独立した数式を取り扱うときは

\begin{sphinxVerbatim}[commandchars=\\\{\}]
\PYG{p}{..}  \PYG{o+ow}{math}\PYG{p}{::}

    \PYGZbs{}frac\PYGZob{}\PYGZbs{}partial \PYGZbs{}rho\PYGZcb{}\PYGZob{}\PYGZbs{}partial t\PYGZcb{} = \PYGZhy{}\PYGZbs{}nabla\PYGZbs{}cdot \PYGZbs{}left(\PYGZbs{}rho \PYGZob{}\PYGZbs{}boldsymbol v\PYGZcb{}\PYGZbs{}right)
\end{sphinxVerbatim}

とすると以下のように表示される。
\begin{quote}
\begin{equation*}
\begin{split}\frac{\partial \rho}{\partial t} = -\nabla\cdot \left(\rho {\boldsymbol v}\right)\end{split}
\end{equation*}\end{quote}

インラインの数式では

\begin{sphinxVerbatim}[commandchars=\\\{\}]
ここで \PYG{n+na}{:math:}\PYG{n+nv}{`\PYGZbs{}rho\PYGZus{}1=x\PYGZca{}2`} とする
\end{sphinxVerbatim}

とすると
\begin{quote}

ここで \(\rho_1=x^2\) とする
\end{quote}

と表示される。

最終更新日:2020年04月28日


\chapter{ライセンス}
\label{\detokenize{index:id1}}
R2D2は現状、公開ソフトウェアではなく再配布も禁じている。
これは今後、変更される可能性はあるが、開発者(堀田)の利益を守るためである。
共同研究者のみが使って良いというルールになっており、R2D2の使用には、以下の規約を守る必要がある。
\begin{itemize}
\item {} 
再配布しない

\item {} 
改変は許されるが、その時の実行結果について堀田は責任を持たない

\item {} 
R2D2で実行する計算は、堀田と議論する必要がある。パラメタ変更などの細かい変更には相談する必要はないが、新しいプロジェクトを開始するときはその都度相談すること。堀田自身のプロジェクト、堀田の指導学生のプロジェクトとの重複を避けるためである。

\item {} 
R2D2を用いた論文を出版するときは \sphinxhref{https://ui.adsabs.harvard.edu/abs/2019SciA....5.2307H/abstract}{Hotta et al., 2019} を引用すること。より詳しく説明した論文も出版予定である。

\item {} 
R2D2を用いた研究を発表するときは、\sphinxhref{https://hottahd.github.io/R2D2-manual/\_images/R2D2\_logo\_red.png}{R2D2のロゴ} の使用が推奨される(強制ではない)。

\end{itemize}


\chapter{TODOリスト}
\label{\detokenize{index:todo}}

\chapter{索引と検索ページ}
\label{\detokenize{index:id3}}\begin{itemize}
\item {} 
\DUrole{xref,std,std-ref}{genindex}

\item {} 
\DUrole{xref,std,std-ref}{search}

\end{itemize}

最終更新日:2020年04月28日


\renewcommand{\indexname}{Pythonモジュール索引}
\begin{sphinxtheindex}
\let\bigletter\sphinxstyleindexlettergroup
\bigletter{r}
\item\relax\sphinxstyleindexentry{R2D2}\sphinxstyleindexpageref{io:\detokenize{module-R2D2}}
\end{sphinxtheindex}

\renewcommand{\indexname}{索引}
\printindex
\end{document}